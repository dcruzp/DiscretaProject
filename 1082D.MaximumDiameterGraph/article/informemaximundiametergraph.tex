\documentclass[10pt]{article}  

\usepackage[spanish,es-noshorthands]{babel}
\usepackage{tikz}
\usepackage{minted}
\usepackage[hidelinks]{hyperref} 
\usepackage{color}
\usepackage{amsmath}
\usepackage{algpseudocode}
\usepackage{algorithm}
\usepackage{makeidx} 

\definecolor {bg} {rgb}{0.95,0.95,0.95}

\usetikzlibrary{positioning,arrows.meta}
\usetikzlibrary{shapes,snakes}

\begin{document}
    \title{1082 - Maximum Diameter Graph}
    \author{Daniel de la Cruz Prieto}
    \date{\today}
    \maketitle

    \begin{abstract}
        \noindent En este art\'iculo se hace un analisis completo , 
        algoritmico para la solucion del ejercicio del Juez en 
        l\'inea \href{https://codeforces.com}{\textcolor{blue}{CodeForce}} 
        si quiere entrar directamente al link del problema 
        \href{https://codeforces.com/problemset/problem/1082/D}{\textcolor{blue}{click aqui}} 
        . En este articulo se muestra un analisis algoritmico , de tiempo y correctitud del 
        algoritmo usado para resolver el problema 
    \end{abstract}

    \section*{Descripci\'on del Problema } 
    
    \begin{flushleft}
        Los problemasconstructivos de gr\'aficos est\'an de vuelta!Esta vez, el gr\'afico que se le pide que cree debe coincidir con las siguintes propiedades.
    \end{flushleft}

    \begin{flushleft}
       El gr\'fico est\'a conectado si y solo si existe una ruta entre cada par de v\'ertices.
    \end{flushleft}

    \begin{flushleft}
        El di\'ametro (tambi\'en conocido como "camino m\'as largo y corto") de un gr\'afico no dirigido conectado es el n\'umero maximo de aristas en el camino {\bf m\'as corto} entre cualquier par de sus v\'ertices.
    \end{flushleft}

    \begin{flushleft}
       El grado de un v\'ertice es el n\'umero de aristas que le inciden.
    \end{flushleft}

    \begin{flushleft}
        Dada una secuencia de $n$ enteros $a_1,a_2,\dots,a_n$ construya una gr\'afica {\bf no dirigida conectada} de v\'ertices $n$ tal que:$n$ enteros $a_1,a_2,\dots,a_n$ construir un $n$ v\'ertices tales que:
    \end{flushleft}

    \begin{itemize}
        \item El gr\'afico no contiene bucles ni bordes m\'ultiples;
        \item El grado $d_i$ del $i$ -th v\'ertices no excede $a_i$ (es decir, $d_i\le a_i$);$d_i$ del $i$-th v\'ertice no exeda $a_i$(es decir $d_i \neq a_i$);
        \item El di\'ametro del gr\'afico es el m\'aximo posible.
    \end{itemize}
    
    \begin{flushleft}
       Genere el gr\'avico resultante o informe q no existe una soluci\'on.
    \end{flushleft}

    \begin{flushleft}
        {\bf Especificaci\'on de la entrada }
    \end{flushleft}

    \begin{flushleft}
       La primera l\'inea contiene un solo entero $n(3\leq n \leq 500) $ el numero de v\'ertices del gr\'afico. 
    \end{flushleft}

    \begin{flushleft}
        La segunda l\'inea contiene n integers $a_1,a_2,\dots,a_n$($1\leq a_i \leq n-1$) los l\'imites superiores a los grados de v\'ertice.
    \end{flushleft}

    \begin{flushleft}
        {\bf Especificaciones de la Salida}
    \end{flushleft}

    \begin{flushleft}
        Imprima "NO" si no se puede contruir un gr\'afico en las condiciones dadas.
    \end{flushleft}
    
    \begin{flushleft}
        De lo contrario, imprima "SI" y el di\'ametro del gr\'afico resultante en la primera l\'inea.
    \end{flushleft}
    
    \begin{flushleft}
        La segunda l\'inea debe contener un solo entero $m$- el n\'umero de aristas en el gr\'afico resultante.
    \end{flushleft}

    \begin{flushleft}
        Los $i-esimos$ del siguiente $m$ las l\'ineas deben contener dos enteros $v_i,u_i(1\leq v_i,u_i\leq  n, v_i \neq u_i)$-la descripci\'on del $i-th$ borde. El gr\\'afico no debe contener m\'ultiples aristas, para cada par($x,y$) emites,no deber\'ias generar m\'as pares ($x,y$) o ($y,x$)
    \end{flushleft}

    \noindent \textbf {Ejemplo de entrada 1} 

    \begin{minted}[bgcolor = bg , frame = single , framerule = 1pt , framesep = 5pt , gobble = 8 , label = INPUT] {console}
        3
        2 2 2
    \end{minted}

    \noindent \textbf {Ejemplo de salida 1} 
    \begin{minted}[bgcolor = bg , frame = single , framerule = 1pt , framesep = 5pt , gobble = 8 , label = OUTPUT] {console}
       SI 2
       2 
       1 2
       2 3
    \end{minted}

    \noindent \textbf {Ejemplo de entrada 2} 

    \begin{minted}[bgcolor = bg , frame = single , framerule = 1pt , framesep = 5pt , gobble = 8 , label = INPUT] {console}
        5
        1 4 1 1 1
    \end{minted}

    \noindent \textbf {Ejemplo de salida 2} 
    \begin{minted}[bgcolor = bg , frame = single , framerule = 1pt , framesep = 5pt , gobble = 8 , label = OUTPUT] {console}
       SI 2
       4
       1 2
       3 2
       4 2
       5 2
    \end{minted}

    \noindent \textbf {Ejemplo de entrada 3} 

    \begin{minted}[bgcolor = bg , frame = single , framerule = 1pt , framesep = 5pt , gobble = 8 , label = INPUT] {console}
        3
        1 1 1 
    \end{minted}

    \noindent \textbf {Ejemplo de salida 3} 
    \begin{minted}[bgcolor = bg , frame = single , framerule = 1pt , framesep = 5pt , gobble = 8 , label = OUTPUT] {console}
        NO
    \end{minted}


    \section{Ideas para resolver el Problema } 

    \noindent Lo que se nos esta pidiendo en el problema es formar un grafo conexo , no dirigido donde el di\'ametro sea el mayor posible 
    pero por el resultado visto en la primera conferencia , el menor grafo conexo que se puede armar es un \'arbol por lo que la suma de los degree tiene que ser mayor o igual  que $2\left(n+1\right)$  
    \\*
    multiplicamos por dos porque cada arista aporta un degree a cada vetice y como el grafo es no dirigido entonces cada arista aporta degree 2 a la suma total de degree del grafo que se quiere armar  

    \vspace*{0.5cm}
    Despues tendr\'imos que tratar de armar un grafo donde  el diametro fuera maximo 
    \\*
    Si se toman los vertices de grado mayor que uno y los unimos a todo en una dadena entonces estar\'iamos maximizando el diam\'etro . Si existen dos vertices de grado uno entonces los pongo a los extremos de la cadena que voy formando   
    \noindent 


    \section{Seudocodigo del Algoritmo } 

    \begin{algorithm}[H]
        \caption{\textit{Dikjstra's algorithm}}
        \begin{algorithmic}[1]      
        \State s $ \leftarrow$ 0 
        
        \While {$i < n $}
                \State s = s + $a\left[i\right]$
                \State i = i + 1 
        \EndWhile 

        \If{$s < 2\left(n-1\right)$}
            \State \textbf{Print "NO"}
            \State \textbf{Return}
        \EndIf
        \State i = 0 
        \While{$i< n$}
            \If{$a\left[i\right] = 1 $}
                \State $a\left[i\right] = 0$
                \State append $\left(ONES , i  \right)$
                \State i = i + 1 
            \EndIf 
        \EndWhile

        \State $t\leftarrow $ ONES.size ()
        \State dm $\leftarrow$ $\left(n-t\right) - 1$ + \textbf{min} $\left(2,t\right)$
        \State PRINT (YES +  dm + n - 1) 

        \State l = -1 
        \If{$ONES.size >0$}
            \State $l$ = $ONES.popback()$
        \EndIf
        
        \State $i = 0$
        \While{$i < n $}
            \If{$a\left[i\right] > 1 $}
                \If{$l \neq -1$ }
                    \State $a\left[l\right] = a\left[l\right]  -1  $
                    \State $a\left[i\right] = a\left[i\right]  -1  $
                \EndIf
                \State l = i 
            \EndIf
        \EndWhile
        \State $i = n-1$
        \While{$i>=0$}
            \While{\textbf{not } ONES.empty() \textbf{and } $a\left[i\right] > 0 $}
                \State $a\left[i\right] = a\left[i\right] - 1 $
                \State PRINT (i+1 , ONES.popback () + 1 )
            \EndWhile
            \State i = i - 1 
        \EndWhile
        \end{algorithmic}
    \end{algorithm}

\end{document}


