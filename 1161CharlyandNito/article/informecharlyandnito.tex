\documentclass{article}

\usepackage[spanish,es-noshorthands]{babel}
\usepackage{tikz}
\usepackage{minted}
\usepackage[hidelinks]{hyperref} 
\usepackage{color}
\usepackage{amsmath}
\usepackage{algpseudocode}
\usepackage{algorithm}

\definecolor {bg} {rgb}{0.95,0.95,0.95}

\usetikzlibrary{positioning,arrows.meta}
\usetikzlibrary{shapes,snakes}



\begin{document}
    \title{1161 - Charly and Nito} 
    \author{Daniel de la Cruz Prieto} 
    \date{\today} 
    \maketitle

    \begin{abstract}
        \noindent En este art\'iculo se hace un analisis completo , algoritmico para la solucion del ejercicio 
        del Juez en l\'inea \href{https://coj.uci.cu}{\textcolor{blue}{Caribbean Online Judge (COJ)}} si quiere entrar directamente al link del 
        problema \href{https://coj.uci.cu/24h/problem.xhtml?pid=1161}{\textcolor{blue}{click aqui}}   . En este articulo se muestra un analisis algoritmico , de tiempo y correctitud del 
        algoritmo usado para resolver el problema 
    \end{abstract}


    \section*{Descripcion del Problema } 

    \begin{flushleft}
    Charly y Nito son amigos y les gusta estar juntos en un lindo bar en 
    Palermo Hollywood. Aproximadamente a las 3 a.m., comienzan a
    sentirse somnolientos y quieren irse a casa.
    Quieren llegar a casa r\'apidamente, por lo que cada uno usa 
    un camino que minimiza la distancia a su casa. Sin embargo,
    a Charly y Nito también les gusta caminar juntos mientras 
    hablan de los "buenos viejos tiempos", por lo que quieren 
    caminar juntos tanto como sea posible. Charly y Nito viven 
    en una ciudad que puede modelarse como un conjunto de 
    calles y cruces. Cada calle conecta un par de cruces 
    distintos y se puede caminar en ambas direcciones. 
    No hay dos calles que conecten el mismo par de cruces. 
    Charly y Nito no viven juntos y no viven en el bar. 
    Hay al menos un camino desde el bar hasta la casa de Charly; 
    lo mismo ocurre con la casa de Nito. 
    Dada la información sobre las calles y cruces de la ciudad, 
    las ubicaciones del bar, la casa de Charly y la casa de Nito,
    debes decirle a Charly y Nito la distancia m\'axima que pueden 
    caminar juntos sin obligarlos a caminar m\'as de la 
    distancia m\'inima desde el bar. a sus respectivos hogares.
    Charly y Nito tambi\'en quieren saber cu\'anto caminar\'a 
    cada uno de ellos solo.
    \end{flushleft}

    \begin{flushleft}
        {\bf Especificaci\'on de la entrada }
    \end{flushleft}
    
    \begin{flushleft}
    La entrada contiene varios casos de prueba, cada uno descrito 
    en varias l\'ineas. La primera l\'inea de cada caso de prueba
    contiene cinco enteros {\bf J, B, C, N y S} separados por espacios
    simples . El valor {\bf J} es el n\'umero de cruces en la ciudad
    {\bf  (3 $\leq$ J $\leq$ 5000)} , cada uni\'on se identifica con un
    n\'umero entero entre {\bf 1 y J} . Los valores {\bf B, C y N} son 
    los identificadores de los cruces donde se encuentran el bar,
    la casa de Charly y la casa de Nito, respectivamente. {\bf (1 $\leq$ B,C,N $\leq$ J);}
    estos tres identificadores de uni\'on son diferentes 
    The value S is the number of streets in the city 
    {\bf (2 $\leq$ S $\leq$ 150000)} .
    Cada una de las siguientes {\bf S} l\'ineas  contiene la descripci\'on de una calle. 
    Cada calle se describe utilizando tres n\'umeros enteros {\bf E1, E2 y L} separados 
    por espacios simples, donde {\bf E1 y E2} identifican dos cruces distintos que son
    puntos finales de la calle {\bf (1 $\leq$ E1, E2 $\leq$ J)}, y L es la longitud de 
    la calle {\bf(1 $\leq$ L $\leq$ $10^4$ )}. Puede suponer que 
    cada calle tiene un par diferente de puntos finales y 
    que existen rutas desde el cruce {\bf B} hasta los cruces {\bf C} y {\bf N}. 
    La \'ultima l\'inea de la entrada contiene 
    el n\'umero $-1$ cinco veces separados por espacios simples y no debe procesarse como
    un caso de prueba. La entrada debe leerse desde la entrada est\'andar. 
    \end{flushleft}


    \begin{flushleft}
        {\bf Especificaciones de la Salida}
    \end{flushleft}

    \begin{flushleft}
        Para cada caso de prueba, la salida de una sola l\'inea 
        con tres enteros {\bf T, C y N } separados por espacios
        simples, donde {\bf T} es la distancia m\'axima que 
        Charly y Nito pueden caminar juntos, {\bf C} es la 
        distancia que Charly camina solo y {\bf N} es la distancia 
        que Nito camina solo. La salida debe escribirse en salida
        est\'andar.
    \end{flushleft}

    \noindent \textbf {Ejemplo de entrada } 

    \begin{minted}[bgcolor = bg , frame = single , framerule = 1pt , framesep = 5pt , gobble = 8 , label = INPUT] {console}
        5 3 2 1 6
        3 4 10
        4 5 10
        5 1 3
        5 2 4
        1 3 23
        2 3 24
        8 1 7 8 8
        1 2 1
        2 4 1
        2 3 1
        4 5 1
        3 5 1
        5 6 1
        6 8 1
        6 7 1
        -1 -1 -1 -1 -1
    \end{minted}
    \noindent \textbf {Ejemplo de salida} 
    \begin{minted}[bgcolor = bg , frame = single , framerule = 1pt , framesep = 5pt , gobble = 8 , label = OUTPUT] {console}
        20 4 3
        4 1 1 
    \end{minted}

    \paragraph*{\large Para resolve este problema } 
    Vamos a debatir lo que nos estan pidiendo en el ejercicio . Primero tenemos  un grafo \textit{G} no dirigido , sin lazos y sin aristas multiples , 
    y dada la informacion de la ubicacion de la casa de Nito , Charly y el Bar (que son distintos los tres) tenemos que encontrar la distancia maxima que pueden caminar Charly y Nito sin que ninguno de los dos camine mas 
    de la distancia minima que hay del bar a sus respectivas casas.
    
    \paragraph*{\large Idea para la resolucion del problema } 
    El primer punto que necesitamos es saber cual es la distancia minima del Bar a las respectivas casas . Hay que encontrar una forma de saber si una interseccion se encuntra en uno de los caminos de longitud m\'inima que va desde el Bar a la casa de alguno de ellos .
    Esto lo podemos hacer iterando por todos los posibles caminos de longitud minima que salen del bar hasta el punto x y chequeamos si en uno de esos caminos se encuentra la interseccion por la que estamos preguntando . 
    \\*
    La soluci\'on seria ir por todas las intersecciones y preguntar si esta es una interseccion que se encuentra en un camino de longitud minima del Bar a la casa de Nito y si a la vez es una interseccion que se encuentra en una camino de longitud m\'inima del Bar a la Casa de Charly , luego nos quedamos 
    con la interseccion que cumpla las condiciones anteriores y que ademas cumpla que su distancia al Bar sea la mayor 
    
    \paragraph*{Cosas necesarias para la resoluci\'on del problema } 
    Necesitamos calcular la distancia minima que hay partiendo de un v\'ertice \textit{B} en un Grafo \textit{G} ponderado hasta todos los demas vertices del Grafo 
    \\*
    Para esto vamos a usar el algoritmo de Dikjstra que es perfectamente aplicable en nuestro problema pues los grafos que se presentan en el problema son grafos no dirigidos ponderados donde el peso de las aristas siempre es positivo 


    \paragraph*{Nota:} vamos a denotar a $d_u\left[v\right]$ como la menor distancia que hay del v\'ertice $u$ al v\'ertice  $v$ 

    \newtheorem{thm}{Teorema}
    \begin{thm}
        En un grafo $G$ ponderado y no dirigido se cumple que $d_u\left[v\right] = d_v\left[u\right] $ 
    \end{thm}

    \noindent Vamos a suponer que el tenemos un grafo $G$ en el que no se cumple que $d_u\left[v\right] ~= ~d_v\left[u\right] $ ,  o sea que existen 
    v\'ertices $u$ y $v$ tal que $d_u\left[v\right] \neq d_v\left[u\right] $ .
    \\[10pt]
    Si  $d_u\left[v\right] \neq d_v\left[u\right] $  es porque $d_u\left[v\right] < d_v\left[u\right] $  o porque $d_u\left[v\right] > d_v\left[u\right] $ 
    \\[10pt]
    Vamos a demostrar que no puede ocurrir que $d_u\left[v\right] < d_v\left[u\right] $
    \\*
    Si $d_u\left[v\right] < d_v\left[u\right] $ es porque existe al menos un camino $c_1 = \{u ,\dots, v\}$ de longitud mininma que va del vertice $u$ al vertice $v$ 
    como el grafo es n dirigido podemos tomar el camino que va del vertice $v$ al vertice $u$ siguiendo la misma secuencia de vertices del camino $c_1$ lo que en el otro sentido ,  es decir 
    tomar a $c_2$ como $\{v,\dots , u\}$ , Este camino $c_2$ que va de $v$ a $u$ tiene la misma longitud que $c_1$ .Por lo que se cumple  $d_v\left[u\right] \leq d_u\left[v\right]$. 
    Luego hemos demostrado que  no puede ocurrir que $d_u\left[v\right] < d_v\left[u\right] $
    \\[10pt]
    Si hacemos un analisis igual podemos demostrar tambien que tampoco puede ocurrir que $d_u\left[v\right] > d_v\left[u\right] $ . 
    Luego no puede ocurrir que $d_u\left[v\right] \neq d_v\left[u\right] $ 
    \\[5pt]
    $\Rightarrow$ Por lo que podemos afirmar que $d_u\left[v\right] ~= ~ d_v\left[u\right] $ en un grafo ponderado y no dirigido 

    \begin{thm}
        Sea $u$ , $v$ dos v\'ertices distintos de un Grafo  $G$ ponderado y no dirigido donde se tiene  que $d_u\left[v\right] = d_v\left[u\right] = m $  , y sea $t$ un v\'ertice de $G$ distinto de $u$ y de $v$ entonces se cumple que si
        $d_u\left[t\right] + d_v\left[t\right] = m $ entonces el v\'ertice $t$ pertenece a un camino m\'inimo entre $u$ y $v$  
    \end{thm}

    Vamos a suponer que $d_u\left[t\right] + d_v\left[t\right] = m $ y que el v\'ertice $t$ no pertenece a ninguno de los caminos minimos de $u$ a $v$ 
    \\[5pt] 
    $\Rightarrow$ Si $d_u\left[t\right]$ se puede calcular es porque existe  al menos un camino de longitud m\'inima del v\'ertice $u$ al v\'ertice  $t$  , vamos a llamrle a este camino $C_1$
    \\*
    Si $d_v\left[t\right]$ se puede calcular es porque existe  al menos un camino de longitud m\'inima del v\'ertice $v$ al v\'ertice  $t$  , vamos a llamrle a este camino $C_2$
    \\*
    Si tomamos el camino $C_1$ y despues el camino $C_2$ podemos ir del vertice $u$ al vertice $u$ pasando por el vertice $t$ . Esto se puede hacer , pues el grafo es no dirigido y ponderado luego se existe un camino de longitud minima del vertice $v$ al vertice $t$ tambien va 
    a existir un camino de longitud m\'inima del v\'ertice $t$ al v\'ertice $v$ y ademas sabemos que $d_t\left[v\right] = d_v\left[t\right]$ por el resultado enterior .
    \\*
    Ahora si llamamos $C$ al camino que va de $u$ a $v$ pasando por $t$ que se puede formar con el camino $C_1$ y el camino $C_2$  . Ahora como se cumple que $d_t\left[v\right] = d_v\left[t\right]$  entonces tenemos que la longitud del camino $C$ esta dada por $d_u\left[t\right] + d_t\left[v\right]$ pero sabemos que esto es igual a 
    $d_u\left[t\right] + d_v\left[t\right] = m $ , Por lo que tenemos que el camino formado $C$ tiene longitud minima . Por lo tanto el camino $C$ es un camino de longitud minima que va del vertice $u$ al vertice $v$ y se cumple que el vertice $t$ pertenece al camino $C$ . Luego hemos encontrado un camino de longitud m\'inima que va de $u$ a $v$ pasando por $t$ .
    \\*
    Luego he encontrado una contradiccion con lo que habia supuesto . 
    \\* 
    Entonces podemos concluir que si se cumple que $d_v\left[t\right] + d_v\left[t\right] = m = d_u\left[v\right] ~=~ d_v\left[u\right]$    el vertice $t$ pertenece a un camino de longitud m\'inima    
    
    
    \section*{Algoritmo de Dijkstra}

    \begin{algorithm}[h]
        \caption{\textit{Dikjstra's algorithm}}
        \begin{algorithmic}[1]      
          \State $\lambda \left(s \right) \leftarrow 0 $
          \State $Q \leftarrow \{s\}$
          \State $P \leftarrow \emptyset $

          \While {$Q \neq \emptyset $}
                \State \textbf{choose a vertex $v \in Q $ for which $\lambda \left(v\right)$ is minimum}
                \State $Q$ $\leftarrow$ $Q$ $\backslash$ $\{v\}$
                \State $P$ $\leftarrow$ $P$ $\cup$  $\{v\}$
                \ForAll {$<u,v>$}
                    \If {$u \in Q $}
                        \State $\lambda $ $\left(u\right)$ $\leftarrow $ \textbf{min} $\{ \lambda \left(u\right) , \lambda \left(v\right) + l \left(e\right)\}$ 
                    \Else
                        \If {$u \notin P$} 
                            \State $\lambda\left(u\right)$ $ \leftarrow $ $ \lambda \left(v\right) + l \left(e\right)$
                            \State $ Q \leftarrow  Q \cup \{u\}$
                        \EndIf
                    \EndIf 
                \EndFor
          \EndWhile 
        \end{algorithmic}
    \end{algorithm}

    
    \subsection*{Lema 1}
        \textit{Cuando al algoritmo de Diskjstra se aplica sobre un grafo finito $G\left(V,E\right)$ este termina} 

    \textbf{Demostracion } 
        Este algoritmo trabaja basicamente sobre la cola $Q$ de vertices 
        Cada vertice $v$ entra a la cola $Q$ a lo sumo una vez , esto lo podemos ver claramente en la 14 donde primero se pregunta si el vertice a incluir en 
        la cola $Q \notin P $ .Aqui la relacion de los conjuntos $P$ y $Q$ es que una vez que un vertice es escogido como el de menor "peso" este sale de la cola 
        $Q$ y pasa a ser parte del conjunto $P$ . 
        \\*
        Luego una vez que un vertice es escogido en la linea 5 este sale inmediatamente de la cola $Q$ y entra en el conjunto $P$ . 
        \\*
        Luego un vertice $v$ no puede entrar mas de una vez a $Q$ . por lo que una vez que la linea 5 se haya ejecutado $|V| $ veces (donde $V$  es el conjunto de vertices del grafo $G$)
    
    \subsection*{Lema 2} 
        \textit{Durante la ejecucion del algoritmo de Dijkstra cada v\'ertice accesible desde $s$ queda marcado , ($\lambda (v)$toma un valor) }

\end{document}
