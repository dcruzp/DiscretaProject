\documentclass[10pt]{article}  

\usepackage[spanish,es-noshorthands]{babel}
\usepackage{tikz}
\usepackage{minted}
\usepackage[hidelinks]{hyperref} 
\usepackage{color}
\usepackage{amsmath}
\usepackage{algpseudocode}
\usepackage{algorithm}
\usepackage{makeidx} 

\definecolor {bg} {rgb}{0.95,0.95,0.95}

\usetikzlibrary{positioning,arrows.meta}
\usetikzlibrary{shapes,snakes}



\begin{document}
    \title{1161 - Charly and Nito} 
    \author{Daniel de la Cruz Prieto} 
    \date{\today} 
    \maketitle
    

    \begin{abstract}
        \noindent En este art\'iculo se hace un analisis completo , 
        algoritmico para la solucion del ejercicio del Juez en 
        l\'inea \href{https://coj.uci.cu}{\textcolor{blue}{Caribbean Online Judge (COJ)}} 
        si quiere entrar directamente al link del problema 
        \href{https://coj.uci.cu/24h/problem.xhtml?pid=1161}{\textcolor{blue}{click aqui}} 
        . En este articulo se muestra un analisis algoritmico , de tiempo y correctitud del 
        algoritmo usado para resolver el problema 
    \end{abstract}


    \section*{Descripci\'on del Problema } 

    \begin{flushleft}
    Charly y Nito son amigos y les gusta estar juntos en un lindo bar en 
    Palermo Hollywood. Aproximadamente a las 3 a.m., comienzan a
    sentirse somnolientos y quieren irse a casa.
    Quieren llegar a casa r\'apidamente, por lo que cada uno usa 
    un camino que minimiza la distancia a su casa. Sin embargo,
    a Charly y Nito también les gusta caminar juntos mientras 
    hablan de los "buenos viejos tiempos", por lo que quieren 
    caminar juntos tanto como sea posible. Charly y Nito viven 
    en una ciudad que puede modelarse como un conjunto de 
    calles y cruces. Cada calle conecta un par de cruces 
    distintos y se puede caminar en ambas direcciones. 
    No hay dos calles que conecten el mismo par de cruces. 
    Charly y Nito no viven juntos y no viven en el bar. 
    Hay al menos un camino desde el bar hasta la casa de Charly; 
    lo mismo ocurre con la casa de Nito. 
    Dada la información sobre las calles y cruces de la ciudad, 
    las ubicaciones del bar, la casa de Charly y la casa de Nito,
    debes decirle a Charly y Nito la distancia m\'axima que pueden 
    caminar juntos sin obligarlos a caminar m\'as de la 
    distancia m\'inima desde el bar. a sus respectivos hogares.
    Charly y Nito tambi\'en quieren saber cu\'anto caminar\'a 
    cada uno de ellos solo.
    \end{flushleft}

    \begin{flushleft}
        {\bf Especificaci\'on de la entrada }
    \end{flushleft}
    
    \begin{flushleft}
    La entrada contiene varios casos de prueba, cada uno descrito 
    en varias l\'ineas. La primera l\'inea de cada caso de prueba
    contiene cinco enteros {\bf J, B, C, N y S} separados por espacios
    simples . El valor {\bf J} es el n\'umero de cruces en la ciudad
    {\bf  (3 $\leq$ J $\leq$ 5000)} , cada uni\'on se identifica con un
    n\'umero entero entre {\bf 1 y J} . Los valores {\bf B, C y N} son 
    los identificadores de los cruces donde se encuentran el bar,
    la casa de Charly y la casa de Nito, respectivamente. {\bf (1 $\leq$ B,C,N $\leq$ J);}
    estos tres identificadores de uni\'on son diferentes 
    The value S is the number of streets in the city 
    {\bf (2 $\leq$ S $\leq$ 150000)} .
    Cada una de las siguientes {\bf S} l\'ineas  contiene la descripci\'on de una calle. 
    Cada calle se describe utilizando tres n\'umeros enteros {\bf E1, E2 y L} separados 
    por espacios simples, donde {\bf E1 y E2} identifican dos cruces distintos que son
    puntos finales de la calle {\bf (1 $\leq$ E1, E2 $\leq$ J)}, y L es la longitud de 
    la calle {\bf(1 $\leq$ L $\leq$ $10^4$ )}. Puede suponer que 
    cada calle tiene un par diferente de puntos finales y 
    que existen rutas desde el cruce {\bf B} hasta los cruces {\bf C} y {\bf N}. 
    La \'ultima l\'inea de la entrada contiene 
    el n\'umero $-1$ cinco veces separados por espacios simples y no debe procesarse como
    un caso de prueba. La entrada debe leerse desde la entrada est\'andar. 
    \end{flushleft}


    \begin{flushleft}
        {\bf Especificaciones de la Salida}
    \end{flushleft}

    \begin{flushleft}
        Para cada caso de prueba, la salida de una sola l\'inea 
        con tres enteros {\bf T, C y N } separados por espacios
        simples, donde {\bf T} es la distancia m\'axima que 
        Charly y Nito pueden caminar juntos, {\bf C} es la 
        distancia que Charly camina solo y {\bf N} es la distancia 
        que Nito camina solo. La salida debe escribirse en salida
        est\'andar.
    \end{flushleft}

    \noindent \textbf {Ejemplo de entrada } 

    \begin{minted}[bgcolor = bg , frame = single , framerule = 1pt , framesep = 5pt , gobble = 8 , label = INPUT] {console}
        5 3 2 1 6
        3 4 10
        4 5 10
        5 1 3
        5 2 4
        1 3 23
        2 3 24
        8 1 7 8 8
        1 2 1
        2 4 1
        2 3 1
        4 5 1
        3 5 1
        5 6 1
        6 8 1
        6 7 1
        -1 -1 -1 -1 -1
    \end{minted}
    \noindent \textbf {Ejemplo de salida} 
    \begin{minted}[bgcolor = bg , frame = single , framerule = 1pt , framesep = 5pt , gobble = 8 , label = OUTPUT] {console}
        20 4 3
        4 1 1 
    \end{minted}

    \section{Analisis del problema} 
    \subsection*{Discuci\'on del problema } 

    \noindent Vamos a hacer un debate de lo que tenemos para la resolucion de este problema. Vamos a analizar 
    que nos dan y que nos piden .
    \\*
    Tenemos un grafo que cumple las siguiente condiciones :

    \begin{itemize}
        \item El grafo $G$ es un grafo no dirigido (Esto quere decir que si existe una arista del v\'ertice $v$ al v\'ertice $u$ entonces tambien va a existir una arista del v\'ertice $u$ al v\'ertice $v$ )
        \item El grafo $G$ no tiene lazos (Esto quiere decir que no puede existir una arista que conecte al mismo v\'ertice)
        \item El grafo $G$ no tiene arista m\'ultiples (esto quiere decir que dos v\'ertices pueden estar conectados por a lo sumo una arista  ) 
        \item El grafo $G$ es ponderado (Esto es que todos las aristas tienen una funcion de dado  asociada )
    \end{itemize}
      

    \noindent Entonces tenemos un grafo no dirigido , sin lazos , sin aristas m\'ultiples y con una funci\'on de peso asociada a sus aristas  
    \\*
    Nos dan tres v\'ertices distintos $N$ , $C$ y $B$ , la ubicaci\'on de la casa de Charly  , de Nito y del Bar respectivamente  
    
    \vspace*{0.5cm}
    \noindent Y nos piden determinar cual es la mayor distancia que pueden caminar Charly y Nito juntos partiendo desde el Bar sin que ninguno de los dos camine mas que la distancia m\'inima que hay del Bar a sus respectivas casas
    \\*
    Usando un lenguaje mas t\'ecnico esto quiere decir que tenemos  que encontrar un v\'ertice $t$ que cumpla las siguientes condiciones : 

    \begin{itemize}
        \item Que el v\'ertice $t$ se encuentre en un camino de costo minimo del Bar a la casa de Charly 
        \item Que el v\'ertice $t$ se encuentre en un camino de costo minimo del Bar a la casa de Nito
        \item Que el v\'ertice $t$ se encuentre lo m\'as alejado posible del v\'ertice $B$ 
        \item Que existan dos caminos $c_1$ y $c_2$ de costo m\'inimo del v\'ertice $B$ a los v\'ertices $N$ y $C$ donde los camino $c_1$ y $c_2$ sean distintos despues del v\'ertice $t$ (que hasta el v\'ertice $t$ los caminos sean iguales)  
    \end{itemize}


    \subsection*{ Idea para la resolucion del problema } 
    \begin{itemize}
        \item \textbf{Idea 1 }Lo primero que se me ocurri\'o y que creo que es lo primero que se le ocurre a todo el mundo , es encontrar todos los caminos de costo m\'inino del Bar a la casa de Charly (Conjunto \textit{MCCharly})y todos los caminos de costo m\'inimos del Bar a la casa de Nito (conjunto \textit{MCNito})y por cada uno de los caminos de \textit{CMCharly} probar todos 
        los caminos de \textit{CMNito} y quedarme con el par de caminos que cumpla todas las condiciones que se exponen en el problema .
        \\*
        Si bien sabemos que este procedimiento es v\'alido y que nos va a 
        llevar a la respuesta que queremos tiene como inconveniente que 
        para un vol\'umen de datos de la entrada moderadamente "grande" 
        se nos hace muy engorroso y poco factible. 
        
        \item \textbf{Idea 2} Esta idea es una poco mas acabada y se me ocurri\'o despues de mucho pensar como se podia  llegar a una soluci\'on  (pensanado en terminos computacionales ) 
        y la idea se basa en el uso del algoritmo de Dijkstra para encontrar los caminos de costa m\'inimo . La idea es la siguiente :
        \\*
        Voy a determinar la distancia m\'inima desde el Bar al resto de los v\'ertices del Grafo $G$ y lo voy a guardar en un array , y voy a hacer lo mismo con los vertices $C$ (casa de Charly ) y $N$ (casa de Nito) . Luego con esta informaci\'on voy a tratar de encontrar lo que me piden de la siguiente manera :
        Voy a recorrer todos los vertices del grafo y le voy a preguntar si el pertenece a un camino m\'inimo entre el Bar y la casa de Charly , y despues voy a preguntarle si este se encuentra en un camino minimo del Bar a la casa de Nito .
        Luego me voy a quedar con el vertice que  cumpla la condicion anterior y que ademas se encuntre lo mas distante posible del vertice $B$ (el Bar) .    
        \\*
        Esta idea es la que trate de desarrollar y es la que voy a abordar en este art\'iculo 
    \end{itemize}
    
    
    \section{Soluci\'on al problema } 

    \noindent La soluci\'on a este problema se basa en la segunda idea que expuse anteriormente .
    Pero esta idea necesita fundamentaci\'on , demostraci\'on y an\'alisis de un algoritmo que me compute la soluci\'on 
    
    
    \subsection{Explicaci\'on de la soluci\'on }
    \noindent Voy a argumetar mejor la idea 2 y detallar paso por paso que es lo que quiero hacer 
    
    \begin{enumerate}
        \item Lo primero que voy a hacer es determinar la menor distancia que hay desde $B$ (el Bar) hasta los demas v\'ertices del grafo y lo voy a almacenar en un array $d_{B} $ 
        \item Despu\'es voy a determinar la menor distancia  que hay desde $C$ (la casa de Charly ) hasta los dem\'as v\'ertices del grafo y lo voy a almacenar en un array $d_{C}$
        \item Despu\'es voy a determinar la menor distancia  que hay desde $N$ (la casa de Nito ) hasta los dem\'as v\'ertices del grafo y lo voy a almacenar en un array $d_{N}$
        \item Por \'ultimo voy a recorrer todos los v\'ertices del grafo , y por cada v\'ertice voy a verificar si ese v\'ertice esta en un camino de longitud m\'inima desde el Bar hasta la casa de Charly y si tambi\'en esta en un camino de longitud m\'inima desde el Bar a la casa de Nito  y me voy a quedar con el que cumpla las condiciones anteriores y adem\'as se encuentre mas alejado del Bar   
    \end{enumerate}
     
    \subsection{Como voy a hacer esto } 

    \noindent Para los primeros tres puntos de arriba voy a aplicar el algoritmo de Dikjstra , el cual es un algoritmo que computa las distancias de los caminos de longitud m\'inima partiendo desde un v\'ertice $s$ hasta los demas v\'ertices del grafo 
    \\*
    En el ultimo punto necesito hacer tres preguntas , por cada uno de los v\'ertices del grafo para esto necesito establecer dos cosas : 
    
    \begin{enumerate}
        \item Como saber si un vertice $t$ pertenece a un camino de longitud m\'inima entre un v\'ertice $a$ y otro v\'ertice $b$ 
        \item Saber cual es la menor distancia desde un v\'ertice $a$ hasta otro v\'ertice $b$ 
    \end{enumerate}

    \noindent Saber cual es la menor distancia desde un v\'ertice $a$ hasta otro $b$ se puede hacer auxiliandote en el c\'alculo del algoritmo de Dijkstra 
    \\*
    Ahora lo m\'as interesante es el primer punto : 
    \\*
    \textit{ ?` Como saber si un v\'ertice t pertenece o no a un camino de longitud m\'inima entre un v\'ertice u y un v\'ertice v ?} 
    
    \subsubsection{?`Como saber si un vertice \textit{t} pertenece a un camino de longitud m\'inima entre un v\'ertice \textit{u} y un v\'ertice \textit{v} ?}
    
    \textbf{Nota:} 
    \begin{itemize}
        \item El resultado que vamos a obtener es sobre grafos ponderados no dirigidos  (porque nos interesa solo los grafos ponderado no dirigido para el an\'alisis de este problema  ) 
        \item vamos a denotar a $d_u\left[v\right]$ como la longitud de los caminos m\'inimos que hay del v\'ertice $u$ al v\'ertice  $v$ .Aqui hay que aclarar que si $v$ no es alcanzable desde $u$ es porque no existe ning\'un camino que vaya de $u$ a $v$ , luego es imposible poder calcular $d_u\left[v\right]$ , en este caso asumimos que $d_u\left[v\right] = \infty $
    \end{itemize}
    

    \newtheorem{thm}{Teorema}

    \begin{thm}
        Sea $G$ un Grafo ponderado y no dirigido entonces para todo $u$ , $v$ v\'ertices del Grafo  se cumple que $d_u\left[v\right] = d_v\left[u\right] $ 
    \end{thm}

    \noindent \textit{\textbf{Demostraci\'on}}

    \noindent Vamos a suponer que tenemos un grafo $G$ ponderado y no dirigido en el que existen $u$ , $v$ tal que  $d_u\left[v\right] \neq d_v\left[u\right] $ .
    \\[10pt]
    Si  $d_u\left[v\right] \neq d_v\left[u\right] $  es porque $d_u\left[v\right] < d_v\left[u\right] $  o porque $d_u\left[v\right] > d_v\left[u\right] $ 
    \\[10pt]
    Vamos a demostrar que no puede ocurrir que $d_u\left[v\right] < d_v\left[u\right] $ \textit{Aqui hay que aclarar que es importante que se pueda calcular $d_u\left[v\right]$ si no es asi entonces no tendr\'ia sentido el an\'alisis que estamos haciendo }
    \\[10pt]
    Si $d_u\left[v\right] \neq \infty$ es porque existe al menos un camino m\'inimo $c_1 = \{u ,\dots, v\}$ que va del v\'ertice $u$ al vertice $v$ . 
    Como el grafo es no dirigido podemos tomar el camino que va del v\'ertice $v$ al v\'ertice $u$ siguiendo la misma secuencia de vertices del camino $c_1$ lo que en el otro sentido ,  es decir 
    tomar a $c_2$ como $\{v,\dots , u\}$ , Este camino $c_2$ que va de $v$ a $u$ tiene la misma longitud que $c_1$ .Por lo tanto $d_{v}\left[u\right]$ se puede calcular porque existe al menos un camino de $v$ a $u$  cuya longitud es igual a  $d_{u}\left[v\right]$ . Entonces se cumple que  $d_v\left[u\right] \leq d_u\left[v\right]$. 
    Entonces hemos demostrado que  no puede ocurrir que $d_u\left[v\right] < d_v\left[u\right] $
    \\[10pt]
    Si hacemos un an\'alisis igual al anterior pero con $d_{v}\left[u\right]$ entonces vemos que tampoco puede ocurrir que $d_u\left[v\right] > d_v\left[u\right] $ .  Luego no puede ocurrir que existan v\'ertices $u$ y $v$ tal que $d_u\left[v\right] \neq~d_{v}\left[u\right] $ 
    \\[5pt]
    $\Rightarrow$ Luego en un grafo ponderado y no dirigido  $d_u\left[v\right] = d_v\left[u\right] $ .

    \begin{thm}
        Sea $G$ un Grafo ponderado y no dirigido y sean $u$ , $v$ dos v\'ertices distintos de $G$ donde se tiene  que $d_u\left[v\right] = d_v\left[u\right] = m $  (que $u$ y $v$ son alcanzables mutuamente ), y sea $t$ un v\'ertice de $G$ distinto de $u$ y de $v$ entonces se cumple que si
        $d_u\left[t\right] + d_v\left[t\right] = m $ entonces el v\'ertice $t$ pertenece a un camino m\'inimo entre $u$ y $v$ . 
    \end{thm}

    \noindent \textit{\textbf{Demostraci\'on}}

    \noindent Vamos a suponer que $d_u\left[t\right] + d_v\left[t\right] = m $ y que el v\'ertice $t$ no pertenece a ninguno de los caminos m\'inimos entre $u$ y $v$ 
    \\[5pt] 
    Si $d_u\left[t\right]$ se puede calcular es porque existe  al menos un camino de longitud m\'inima del v\'ertice $u$ al v\'ertice  $t$  , vamos a llamarle a este camino $C_1$.
    Igual ocurre si  $d_v\left[t\right]$ se puede calcular es porque existe  al menos un camino de longitud m\'inima del v\'ertice $v$ al v\'ertice  $t$  , vamos a llamarle a este camino $C_2$
    \\*
    Si tomamos el camino $C_1$ y despu\'es el camino $C_2$ podemos ir del v\'ertice $u$ al v\'ertice $v$ pasando por el vertice $t$ . Esto se puede hacer , pues el grafo es no dirigido , luego se existe un camino de longitud m\'inima del v\'ertice $v$ al v\'ertice $t$ tambien va 
    a existir un camino de longitud m\'inima del v\'ertice $t$ al v\'ertice $v$ y adem\'as sabemos que $d_t\left[v\right] = d_v\left[t\right]$ por el resultado anterior .
    \\*
    Ahora si llamamos $C$ al camino que va de $u$ a $v$ pasando por $t$ que se puede formar con el camino $C_1$ y el camino $C_2$  . Como se cumple que $d_t\left[v\right] = d_v\left[t\right]$  entonces tenemos que la longitud del camino $C$ esta dada por $d_u\left[t\right] + d_t\left[v\right]$ entonces 
    $d_u\left[t\right] + d_v\left[t\right] = m $ . Por lo tanto el camino $C$ es un camino de longitud m\'inima que va del v\'ertice $u$ al vertice $v$ y se cumple que el vertice $t$ pertenece al camino $C$ . Luego hemos encontrado un camino de longitud m\'inima que va de $u$ a $v$ pasando por $t$ .
    \\*
    Por lo que existe una contradiccion con lo que habia supuesto . Entonces podemos concluir que si se cumple que $d_v\left[t\right] + d_u\left[t\right] = m $    el v\'ertice $t$ pertenece a un camino de longitud m\'inima  entre $u$ y $v$  
    
    \subsubsection{Seudocodigo de la Soluci\'on }
    
    \begin{algorithm}[h]
        \caption{\textit{Solution}}
        \begin{algorithmic}[1]  
            \State $d_{B}$ $\leftarrow shortedPath \left(B\right)$   
            \State $d_{C}$ $\leftarrow shortedPath \left(C\right)$ 
            \State $d_{N}$ $\leftarrow shortedPath \left(N\right)$ 
            \State $dC \leftarrow $ $d_{B}\left[C\right]$
            \State $dN \leftarrow $ $d_{B}\left[N\right]$
            \State $m \leftarrow 0 $
            \State $c \leftarrow dC$
            \State $n \leftarrow dN$ 
            \ForAll {$v \in V\left(G\right)$}
                \If {$dC=d_{B}\left[v\right] + d_{C} \left[v\right]  $ \textbf{and} $ dN = d_{B}\left[v\right] + d_{N} \left[v\right]  $ \textbf{and} $d_{B}\left[v\right] > m $}
                    \State $ m \leftarrow d_{B} \left[v\right]$
                    \State $ c \leftarrow d_{C} \left[v\right]$ 
                    \State $ n \leftarrow d_{N} \left[v\right]$ 
                \EndIf 
            \EndFor
        \end{algorithmic}
    \end{algorithm}

    \noindent Voy a explicar el seudoc\'odigo del algoritmo 
    \\*
    En las lineas 1-3 se aplica el algoritmo de Dijkstra para calcular las menores distancias desde el $B$ (Bar) , $C$ (la casa de Charly) , $N$ (casa de Nito) a los demas v\'ertices del grafo . 
    \textit{sortedPath()} es el algoritmo de Dijkstra cuyo calculo de los caminos minimos es almacenado en los array $d_B$ , $d_C$ , $d_N$   , se denota a $m$ como la mayor distancia que pueden recorrer Charly y Nito juntos  , y $c$ y a $n$ como la distancia que recorre Chraly y Nito solos respectivamente 
    .De la linea 6-8  las variables m , c , n toman valores 0 , distancia m\'inima desde el Bar hasta casa de Charly y distancia m\'inima desde el Bar hasta casa de Nito respectivamente . ahora de la Linea 9-14, el algoritmo lo que hace es recorrer todos los v\'ertices del grafo y preguntar en la linea 10 si 
    el v\'ertice que en el que est\'a  cumple las condiciones expuestas arriba  (esto ma grarantiza que que el v\'ertice $v$ se encuentra en un camino minimo del Bar a la Casa de Nito y tambien que esta en un camino de longitud minima del Bar a la casa de Charly), si adem\'as de estas condiciones cumple que  
    la distancia desde B hasta $v$ cumple que la menor distancia desde $B$ (Bar) hasta $v$ es mayor que $m$ entonces actualizo m , c y n . porque en este punto del algoritmo encontre un v\'ertices  que esta mas alejado del Bar el cual es com\'un a dos de los caminos de longitud minima desde el $B$ hasta $C$ y desde $B$ hasta $N$
    \\[10pt] 
    Pero aqui no estamos teniendo en cuenta una de las condiciones que el problema nos esta plantenado y es que nosotros hemos visto que un v\'ertice $t$ que se encuentre en un camino $C_{min} = \{b,\dots,c\}$  que va del v\'ertice  $B$ al  v\'ertice $C$ y que $t$ se encuentra en un camino m\'inimo $N_{min} = \{b,\dots,n\}$  que va del v\'ertice  $B$ al  v\'ertice $N$ pero 
    no garantizamos que los camino $C_{min}$ y $N_{min}$ sea caminos comunes hasta el vertice $t$ (es decir que el subcamino  $C_{min} $ que va desde $B$ hasta $t$ no tiene porque coincidir con el subcamino $N_{min} $ que va desde $B$ hasta $t$ ).
    \\[10pt] 
    Ahora vamos a puntualizar este aspecto del an\'alisis que estamos plantenado de la soluci\'on al problema 
    \\[10pt]
    \textit{?` Porque no es necesario que el subcamino $C_{min}$ coincida con el subcamino $N_{min}$ para que la soluci\'on sea correcta ?}
    \\[10pt] 
    Vamos a demostrar que si $t$ se encuentra en un camino de longitud m\'inima desde $B$ hasta $N$ y tambi\'en se encuentra en un camino de longitud m\'inima desde $B$ hasta $C$  entonces existen dos caminos  de longitud m\'inima , $C1_{min}$que va de $B$ hasta $C$ y  $N1_{min}$que va desde $B$ hasta $N$ a los cuales $t$ pertenece y que los subcaminos de $N1_{min}$ y de $C1_{min}$ que van desde $B$ hasta $t$ coinciden .

    \begin{thm}
        Sea $G$ un Grafo y $B$ , $N$ y $C$ v\'ertices distintos  del Grafo entonces si existe los camino m\'inimo $C1$ y $N1$ que van de $B$ a $C$ y de $B$ a $N$ a los cuales $t$ pertenece entonces existen tambi\'en dos camino m\'inimos $C_{min}$ y $N_{min}$ a los cuales $t$ pertenece y los subcaminos de $C_{min}$ y de $N_{min}$ que van de $B$ hasta $t$ son iguales  
    \end{thm}

    \noindent \textit{\textbf{Demostraci\'on}}

    \noindent Si los subcaminos que van de $B$ a $t$ en $C1$ y $N1$  coinciden entonces ya cumple la condici\'on
    \\*
    Primero vamos a denotar como $SC$ al subcamino de $C1$ que va de $B$ a $t$  y $SN$ como es subcamino de $N1$ que va de $B$ $t$ . Entonces vamos a analizar el caso en que $SC$ y $SN$ sean distintos 
    \\*
    Esta claro que $SC$ y $SN$ tienen la misma longitud pues son subcaminos de longitud m\'inima desde $B$ hasta $t$ , sino hubiera una contradiccion con lo antes demostrado . 
    \\*
    Como $SC$ y $SN$ tienen la misma longitud puedo formar $C_{min}$  y $N_{min}$ de la sigiente manera:
    $C_{min}$ va a ser el camino $SC$ siguiendo despues por el camino $C1$ a partir del vertice $t$ hasta $C$ (que es el propio camino $C1$).
    Ahora $N_{min}$ lo vamos a formar tomando primero el subcamino $SC$ y despu\'es siguiendo por el camino $N1$ a partir del v\'ertice $t$ hasta $N$ ,
    $N_{min}$ es de longitud minima pues $SC$ y $SN$ tiene la misma longitud.
    \\[10pt]
    $\Rightarrow$ Entonces he construido dos caminos de longitud m\'inima $C_{min}$ que va de $B$ a $C$ y $N_{min}$ que va de $B$ a $N$  tal que sus subcaminos hasta el v\'ertice $t$ son iguales  
    \\[10pt] 
    Luego este resultado nos garantiza que siempre que encontremos dos caminos de longitud m\'inima desde $B$ hasta $C$ y desde $B$ hasta $N$ a los cuales $t$ pertenece, entonces vamos a poder encontrar dos caminos de  longitud m\'inima desde $B$ hasta $C$ y desde $B$ hasta $N$ a los cuales $t$ pertenece , cuyos subcaminos desde $B$ hasta $t$ son iguales .


    \subsection{Correctitud del algoritmo} 

    \noindent La solucion pasa por aplicar Dijkstra , luego su correctitud depende de la correctitud del algoritmo de Dijkstra . 
    Ademas con los teoremas 1 , 2 y 3  antes enunciados y demostrados nos grarantiza que la soluci\'on que se presenta da el resultado correcto. 
    Nuestro algoritmo termina , pues el algoritmo de Diskjstra termina , (esto se demuestra en el an\'alisis de este algoritmo)  , ademas como el grafo es 
    finito recorrer todos los v\'ertices en alg\'un momento no habran mas v\'ertices que preocesar (teniendo en cuenta que cada v\'ertice es analizado una sola vez)  . 




    \section{Algoritmo de Dijkstra}

    \noindent El algoritmo de Dijkstra calcula la menor distancia desde un v\'ertice inicial hasta todos los dem\'as vertices del Grafo  

    \subsection{Seudoc\'odigo del algoritmo de Dijkstra} 

        \begin{algorithm}[h]
            \caption{\textit{Dikjstra's algorithm}}
            \begin{algorithmic}[1]      
            \State $\lambda \left(s \right) \leftarrow 0 $
            \State $Q \leftarrow \{s\}$
            \State $P \leftarrow \emptyset $

            \While {$Q \neq \emptyset $}
                    \State \textbf{choose a vertex $v \in Q $ for which $\lambda \left(v\right)$ is minimum}
                    \State $Q$ $\leftarrow$ $Q$ $\backslash$ $\{v\}$
                    \State $P$ $\leftarrow$ $P$ $\cup$  $\{v\}$
                    \ForAll {$<u,v>$}
                        \If {$u \in Q $}
                            \State $\lambda $ $\left(u\right)$ $\leftarrow $ \textbf{min} $\{ \lambda \left(u\right) , \lambda \left(v\right) + l \left(e\right)\}$ 
                        \Else
                            \If {$u \notin P$} 
                                \State $\lambda\left(u\right)$ $ \leftarrow $ $ \lambda \left(v\right) + l \left(e\right)$
                                \State $ Q \leftarrow  Q \cup \{u\}$
                            \EndIf
                        \EndIf 
                    \EndFor
            \EndWhile 
            \end{algorithmic}
        \end{algorithm}

    
    \subsection*{Lema 1}
        \noindent  \textit{Cuando al algoritmo de Diskjstra se aplica sobre un grafo finito $G\left(V,E\right)$ este termina} 

        \vspace{0.5cm} 
        \noindent \textit{\textbf{Demostracion } }
        \\*
        \noindent Este algoritmo trabaja basicamente sobre la cola $Q$ de v\'ertices 
        Cada v\'ertice $v$ entra a la cola $Q$ a lo sumo una vez , esto lo podemos ver claramente en la linea 14 donde primero se pregunta si el v\'ertice a incluir en 
        la cola $Q \notin P $ .Aqui la relacion de los conjuntos $P$ y $Q$ es que una vez que un v\'ertice es escogido como el de menor "peso" este sale de la cola 
        $Q$ y pasa a ser parte del conjunto $P$ . 
        \\*
        Luego una vez que un vertice es escogido en la linea 5 este sale inmediatamente de la cola $Q$ y entra en el conjunto $P$ . 
        \\*
        Luego un v\'ertice $v$ no puede entrar mas de una vez a $Q$ . por lo que una vez que la linea 5 se haya ejecutado $|V|$ veces el algoritmo termina su ejecuci\'on (donde $V$  es el conjunto de v\'ertices del grafo $G$)
    
    \subsection*{Lema 2} 
        \noindent \textit{Durante la ejecuci\'on del algoritmo de Dijkstra cada v\'ertice accesible desde $s$ queda marcado , ($\lambda (v)$toma un valor) }
        
        \vspace{0.5cm}
        \noindent \textit{\textbf{Demostraci\'on}}
        \\*
        Vamos a suponer que existe un v\'ertice $v$ que es accesible desde $s$
        y que el valor de $\lambda\left(v\right)$ nunca cambia 
        \\*
        Como $v$ es accesible desde $s$ entonces existe un camino $\mathcal{P} $
        que va de $s$ a $v$ . vamos a denotar a $u$ como el primer v\'ertice en 
        el camino $\mathcal{P}$ para el cual el valor de $\lambda$ no cambio
        durante la ejecucion del algoritmo . Y ahora vamos a denotar $w \xrightarrow[]{e} u$ 
        tal que  $\lambda\left(w\right)$ cambio . Entonces en alg\'un punto de la 
        ejecucion del algoritmo el v\'ertice $w$ fue procesado eventualmente para
        salir de la cola $Q$ y con esto la arista $w \xrightarrow[]{e} u$  es procesada y por
        lo tanto el valor de $\lambda\left(u\right)$ cambia 
        \\*
        Luego hemos encontrado una contradicci\'on , habiamos supuesto que 
        $\lambda\left(u\right)$ nunca cambi\'o , pero hemos encontrado que en alg\'un 
        punto de la ejecuci\'on del algoritmo su valor cambio 
        \\*
        Con esta proceso como $v$ esta en el camino $\mathcal{P}$ en alg\'un punto 
        su valor cambia , poque sino entonces encontrariamos una contradiccion como 
        la anterior . 
        \\*
        Luego si $v$ es accesible desde $s$ entonces al aplicar el algoritmo de Dijkstra
        el valor de $\lambda\left(v\right)$ cambia.
        
    \subsection*{Lema 3}

        \noindent \textit{En cualquier momento durante la ejecucion del algoritmo de Diskstra si 
        para un vertice $v $ cambia  su valor de $\lambda\left(v\right)$ entonces existe 
        un camino desde $s$ hasta $v$ cuya longitud es $\lambda\left(v\right)$}

        \vspace{0.5cm}
        \noindent\textit{\textbf{Demostraci\'on}}
        \\*
        Vamos a hacerlo por inducci\'on en el momento en que es modificado el valor de~$\lambda$
        \\*
        \noindent La primera asignaci\'on ocurre en la linea 1 , donde $\lambda\left(s\right)$ toma valor $0$ 
        y en efecto podemos ver que existe un camino de longitud $0$ desde el v\'ertice $s$ hasta el propio 
        v\'ertice  $s$ (un camino vac\'io ). 
        \\*
        Ahora en cada asignaci\'on o reasignaci\'on que se le hace a $\lambda$ en el momento $t$ 
        y asumiendo que en todas las reasignaciones anteriores se cumple la hip\'otesis  . 
        \\*
        Vamos a asumir que a un v\'ertice $u$ su valor de $\lambda\left(u\right)$ cambia por primera vez . 
        Entonces el valor asignado a $v$ fue antes del momento $t$  . 
        \\*
        Por hipotesis de inducci\'on existe un camino que va de $s$ a $v$ de longitud $\lambda\left(v\right)$. 
        Si a ese camino le adicionamos la arista $u \xrightarrow[]{e} v $ entonces es un camino desde $s$ hasta $u$ 
        de longitud $\lambda\left(u\right)$
        \\[5pt] 
        Para el caso en que el valor de $u$ es reasignado (o sea que no fue la primera vez que le es
        asignado un valor a $\lambda\left(u\right)$ ) pudemos encontrar el camino de longitud
        $\lambda\left(u\right)$ de la misma manera .
        \\[5pt]
        Vamos a denotar a $\delta \left(v\right)$ como la menor distancia de $s$ a $v$ entonces 
        tenemos que: 
        \begin{equation*}
            \lambda\left(v\right) \geq \delta\left(v\right)
        \end{equation*}

        \vspace{1cm} 

        \newtheorem{thm1}{Teorema}
        \begin{thm1}
            \textit{Cuando el algoritmo de Dijkstra termina para cada v\'ertice $v$ accesible desde $s$ se cumple  que : }
        \end{thm1}

        \begin{equation*}
            \lambda\left(v\right) = \delta\left(v\right)
        \end{equation*}

        \noindent \textit{\textbf{Demostracion}}
        \\* 
        Si v es accesible desde $s$ entonces por lo demostrado 
        anteriormente $\lambda\left(v\right) < \infty$  . Y por el lema 3  tenemos que  
        $\lambda\left(v\right) \geq \delta\left(v\right)$  . Solo falta demostrar que 
        $\lambda\left(v\right) \leq \delta\left(v\right)$
        \\*
        Vamos a hacer la demostraci\'on por inducci\'on en el orden en que los v\'ertices pasan 
        al conjunto $P$ 
        \\*
        Vamos a denotar a $\mathcal{P}$ como un camino m\'inimo de $s$ a $v$ y Vamos a denotar 
        a $t$ como el momento cuando el vertice $v$ es escogido para unirse al conjunto $P$ .
        En este momento $t$ , el vertice $s$ esta en $P$ pero $v$ no lo esta . 
        \\*
        Pero debe existir una arista $u \xrightarrow[]{e}w $ en $\mathcal{P}$ tal que $u \in P $ pero 
        $w \notin P$ 
        \\* 
        Como todas las aristas tienen peso positivo , el subcamino de $\mathcal{P}$ desde $w$ 
        a $v$ tiene longitud positiva entonces : 
        \begin{equation*}
            \delta \left(v\right) \geq \delta\left(w\right)
        \end{equation*}
        Cada subcamino de un camino de longitud m\'inima es de longitud m\'inima . Entonces el 
        subcamino de $\mathcal{P}$ que va desde $s$ hasta $w$  es de longitud m\'inima y su longitud 
        esta dada por la longitud del camino de longitud m\'inima desde $s$ hasta $u$ mas el peso de la 
        arista $e$ ($l\left(e\right)$) , esto es : 
        \begin{equation*}
            \delta\left(w\right) = \delta\left(u\right) + l\left(e\right)
        \end{equation*}
        Tenemos que $u$ se introdujo en $P$ antes del tiempo $t$ , por la hipotesis de induccion 
        $\delta\left(u\right) \geq \lambda\left(u\right)$ .Entonces 
        \begin{equation*}
            \delta\left(u\right) + l\left(e\right) \geq \lambda\left(u\right) + l\left(e\right)
        \end{equation*} 
        Como $u$ entro en $P$ antes del tiempo $t$ todos sus adyacentes fueron examinados, incluyendo
        a la arista $u \xrightarrow[]{e}w$ . Entonces en al momento $t$ , $\lambda\left(w\right)$ fue asignada y por 
        lo tanto 
        \begin{equation*}
            \lambda\left(u\right) + l\left(e\right) \geq \lambda\left(w\right)
        \end{equation*} 
        Sin embargo , en el momento $t$ , $v$ ha sido el escogido para unirse a $P$ , no  $w$ . entonces : 
        \begin{equation*}
            \lambda\left(w\right) \geq \lambda\left(v\right)
        \end{equation*}
        Entonces podemos decir que todas las anteriores afirmaciones implican que en el momento $t$ 
        \begin{equation*}
            \delta\left(v\right) \geq \lambda\left(v\right)
        \end{equation*}

    \subsection{Complejidad Temporal }
        \noindent Durante la ejecuci\'on del algoritmo de Dijkstra cada v\'ertice entra en $Q$ a lo sumo una vez y cada v\'ertice que entra en $Q$ sale 
        una vez que es escogido en la linea 5 . Entonces la ejecuci\'on de la linea 5 se realiza a lo sumo $\vert V \vert$ veces por lo que su complejidad 
        de ejecuci\'on toma $O\left(\vert V \vert\right)$ . La suma total del tiempo de ejecuci\'on de la linea 8 - 14 para cada v\'ertice escogido es de $O\left(\vert E \vert\right)$. 
        No hay aristas examinadas m\'as de una vez . Esto quiere decir que el tiempo de ejecuci\'on del algoritmo de Dijkstra es de $O\left(\vert V \vert ^ 2 + \vert E \vert \right)$
        
        
    \section{Mi implementaci\'on}
    \noindent Primero hice una implementaci\'on en c++ y otra en Python . La implementaci\'on que probe en el juez en line fue la de c++ y la implementaci\'on en Python es una traducci\'on de la implementaci\'on
    en c++  . que si bien no esta mal la implementacion en python , las herraminetas que utilice puede que no den en un tiempo apropiado como la de c++   

    \subsection{Implementaci\'on  en C++ } 
    \noindent En mi implementaci\'on en c++ para el algoritmo de Dijkstra use "set" que despues de hacer una busqueda en internet encontre que su implementacion usa 
    un arbol binario balanceado por altura . Entonces el tiempo de complejidad de operaciones sobre el "set" como insertar , borrar es logaritmica luego el tiempo de 
    complejidad temporal del algoritmo que implemente es  $O\left(\vert E \vert \cdot \log \left(V\right)\right)$  .  

    \begin{minted}[gobble = 8 , breaklines ,bgcolor = bg,frame = single ,linenos = true , label = "Dijkstra's implementation " ]{c++}
        void shortedPath (int start , int distance []) 
        {
            init(distance);  
            Q.insert(make_pair(0,start)); 
            distance[start] = 0;   
            while (! Q.empty())
            {
                Node tmp = * (Q.begin());  
                Q.erase(Q.begin()); 

                int curr = tmp.second ;  
                vector <Node> :: iterator i ;
                for(i = ady[curr].begin() ; i != ady[curr].end() ; ++i)
                {
                    int v = (*i).first ;    
                    int weight = (*i).second;  

                    if (distance[curr] + weight < distance[v])  
                    {
                        if (distance[v] < INF) 
                        {
                            Q.erase(Q.find(make_pair(distance[v] , v))); 
                        } 
                        distance[v] = distance[curr] + weight ; 
                        Q.insert(make_pair(distance[v],v)); 
                    } 
                }
            }
        }
    \end{minted} 


    \noindent La funcion \textit{solution ()} aplica tres veces el algoritmo implementado arriba y despues recorre todos los vertices del grafo . Por lo que 
    la complejidad temporal del algoritmo es $O\left(3 \cdot \vert E \vert \cdot \log \left(V\right)\right)$ + $O\left(\vert V \vert\right) $


    \begin{minted}[gobble = 8 , breaklines ,bgcolor = bg, linenos = true,frame = single, label = "Algorithm Solution"]{c++}
        void solucion()
        {   
            shortedPath(B-1,distanciaB); 
            shortedPath(C-1,distanciaC); 
            shortedPath(N-1,distanciaN);  

            int distC = distanciaB[C-1]; 
            int distN = distanciaB[N-1]; 
            int j = 0 , c =distC , n =distN; 

            for (int i = 0; i < J; i++)
            {                
                if (distC == distanciaB[i]+distanciaC[i] && distN == distanciaB[i] + distanciaN[i] && distanciaB[i] > j)
                {
                    j = distanciaB[i];  
                    c = distanciaC[i];  
                    n = distanciaN[i]; 
                }
            }            
            cout << j << " " << c << " " << n << endl ; 
        }
    \end{minted}

    \noindent Aqui voy a poner el l\'ink de donde esta el c\'odigo completo en mi repositorio  , adem\'as de que el codigo aparece en la carpeta donde mismo se encuentra este informe 
    
    \subsection{Implementaci\'on  en Python }
    
    \noindent Esta implementaci\'on  lejos de estar mal porque es una traduccion exacta del algoritmo que puse arriba  . Si bien puede no ser lo mejor en cuanto a complejidad temporal debido al uso de 
    librerias que no son muy eficientes a la hora de insertar y  eliminar elementos , cosas necesarias para la implementacion del algoritmo de Dijkstra 
    
    \begin{minted}[gobble = 8 , breaklines ,bgcolor = bg,frame = single ,linenos = true , label = "Dijkstra's implementation " ]{Python}
        def dijkstra (inicial ,distancia):
        init(distancia)
        distancia[inicial] = 0
        priority_queue.put((0,inicial))
        
        while not priority_queue.empty():
            top = priority_queue.get()
            actual = top[1]
            vertice_actual = g.obtenerVertice(actual)
            if visitado[actual]: continue 
            visitado[actual] = True

            for adyacente in vertice_actual.obtenerConexiones():
                peso = vertice_actual.obtenerPonderacion(adyacente)

                if not visitado[adyacente.obtenerId()]:
                    relajacion (actual , adyacente.obtenerId() , peso , distancia)    
    \end{minted} 

    \noindent El metodo \textit{solution()} expuesto es el mismo que el implementado en c++ . El c\'odigo es el de abajo 
    
    \begin{minted}[gobble = 8 , breaklines ,bgcolor = bg, linenos = true,frame = single, label = "Algorithm Solution"]{Python}
        def solucion ():

        dijkstra(B-1,distanciaB)
        dijkstra(C-1,distanciaC)
        dijkstra(N-1,distanciaN)
        
        distC = distanciaB[C-1] 
        distN = distanciaB[N-1]
        j = 0 
        c = distC
        n = distN

        for i in range (J) :
            if distC == distanciaB[i] + distanciaC[i] and distN == distanciaB[i] + distanciaN[i] and distanciaB[i] > j :
                j = distanciaB[i]
                c = distanciaC[i]
                n = distanciaN[i] 

        print (str (j) + ' ' + str (c) + ' ' +str(n))
    \end{minted}
\end{document}
