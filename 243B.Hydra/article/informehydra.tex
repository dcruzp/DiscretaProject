\documentclass[12pt]{article}

\usepackage[spanish,es-noshorthands]{babel}
\usepackage{tikz}
\usepackage{minted}

\usepackage{algpseudocode}
\usepackage{algorithm}
\usepackage{verbatim}
\usepackage{listings} 


\definecolor {bg} {rgb}{0.95,0.95,0.95}


\usetikzlibrary{positioning,arrows.meta}
\usetikzlibrary{shapes,snakes}

\renewcommand{\algorithmicrequire}{\textbf{Input:}}
\renewcommand{\algorithmicensure}{\textbf{Output:}}


\begin{document}
    \title{B.Hydra} 
    \author{Daniel de la Cruz Prieto} 
    \date{\today} 
    \maketitle

    \begin{flushleft}
    	Un d\'ia Petya recibi\'o un regalo de cumplea\~nos de su mam\'a, un libro llamado "Leyes y mitos de la Teor\'ia de Grafos". De este libro Petya aprendi\'o acerca de los grafos Hydra.
    \end{flushleft}
    
    \begin{flushleft}
        Un grafo no dirigido es un Hydra; si este tiene una estructura como la mostrada en la figura de abajo.\\
        Esta estructura tiene dos nodos $u$ y $v$ conectados por una arista. Estos son el pecho y el estomago de Hydra
        correspondientemente. El pecho es conectado con h nodos que son la cabeza de Hydra. El est\'omago est\'a conectado con t nodes que son la cola de Hydra.
    \end{flushleft}
      
    \begin{flushleft}
         Note que el Hydra es un \'arbol, que tiene $h+t+2$ nodos
    \end{flushleft}
    
    \begin{center}
        \begin{tikzpicture}[vertex/.style={draw,circle} ]
        \tikzstyle{stylenode} = [vertex ,fill = red ,draw = none ,text = black ,scale = 1]
            
            \node[ stylenode] (1) at (1.6,6) {} ; 
            \node[ stylenode] (2) at (1.4,5) {} ; 
            \node[ stylenode] (3) at (1.3,4) {} ; 
            \node[ stylenode] (4) at (3,5) {$u$} ; 

            \node[ stylenode] (5) at (7,4) {$v$} ; 
            \node[ stylenode] (6) at (9,5) {} ; 
            \node[ stylenode] (7) at (9,4) {} ; 
            \node[ stylenode] (8) at (9,3) {} ; 
            \node[ stylenode] (9) at (8.4,2) {} ; 
            


            \draw [line width = 2] (1) -- (4);
            \draw [line width = 2] (2) -- (4);
            \draw [line width = 2] (3) -- (4);

            \draw [line width = 2] (4) -- (5);

            \draw [line width = 2] (5) -- (6);
            \draw [line width = 2] (5) -- (7);
            \draw [line width = 2] (5) -- (8);
            \draw [line width = 2] (5) -- (9);

            \node[right = 0.5cm,text width = 4cm,font=\footnotesize,scale = 1.5] at (1)
            {
                heads
            };
            \node[right = 1.5cm,text width = 4cm,font=\footnotesize,scale = 1.5] at (4)
            {
                body
            };
            \node[left = 0.5cm,text width = 0.5cm,font=\footnotesize,scale = 1.5] at (6)
            {
                tails
            };

        \end{tikzpicture}
    \end{center}

    \begin{flushleft}
        Tambien, Petya tiene un grafo no dirigido G que consiste de n nodos y m aristas. Petya recibi\'o este como el \'ultimo regalo de cumplea\~nos de su mam\'a. El grafo G no contiene ni multiples aristas ni lazos.
    \end{flushleft}
  
    \begin{flushleft}
        Ahora Petya quiere encontrar un Hydra en el grafo G o estar segura que el grafo no tiene un Hydra.
    \end{flushleft}

    \noindent \textbf{\Large Ejemplo 1}
    \begin{minted}[bgcolor = bg , frame = single , framerule = 1pt , framesep = 5pt , gobble = 8 , label = INPUT] {console}
        9 12 2 3
        1 2
        2 3
        1 3
        1 4
        2 5
        4 5
        4 6
        6 5
        6 7
        7 5
        8 7
        9 1
    \end{minted}

    \begin{minted}[bgcolor = bg , frame = single , framerule = 1pt , framesep = 5pt , gobble = 8 , label = OUTPUT] {console}
        YES
        4 1
        5 6 
        9 3 2
    \end{minted}
    
    \vspace{1cm}
    \noindent \textbf{\Large Ejemplo 2}
    \begin{minted}[bgcolor = bg , frame = single , framerule = 1pt , framesep = 5pt , gobble = 8 , label = INPUT] {console}
        7 10 3 3
        1 2
        2 3
        1 3
        1 4
        2 5
        4 5
        4 6
        6 5
        6 7
        7 5
    \end{minted}
    
    \begin{minted}[bgcolor = bg , frame = single , framerule = 1pt , framesep = 5pt , gobble = 8 , label = OUTPUT] {console}
        NO
    \end{minted}

    \begin{algorithm}[h]
        \caption{Calculo de \textit{first hop}}
        \begin{algorithmic}[1]
       
          % ENTRADA / SALIDA
          \Require{ G , u , v } 
          \Ensure{head , tail}
       
          \State countcommun = 0 
          \ForAll {$a \in Ady \left[v\right] $} sjdfg\EndFor
       
          \For{$i;\ i++;\ |target|$}
            \State{traceroute($IP = IP[i],\ TTL_{initial} = 1$)}
          \EndFor
       
          \For{$i;\ i++;\ |target|$}
            \State{trace = read\_traceroute($IP[i]$)}
            \State{$TTL_{initial} = 1$}
            \For{hop \textbf{in} trace}
              \If{hop \textbf{in} $IP_{found}$}
                \State $TTL_{initial} ++$
              \Else
                 \State $IP_{found}$ \textbf{append} hop
                 \State $first\_hop[i] = TTL_{initial}$
                 \State \textbf{break}
              \EndIf
            \EndFor
          \EndFor
          \State{\textbf{return} $first\_hop$}
        \end{algorithmic}
      \end{algorithm}

    
    \paragraph*{Idea basica para resolver el algoritmo } 
    Tenemos que dado un grafo G y dos entros h y t determinar si 
    existe alguna estuctura de la forma descrita en la orden del ejercicio. 
    Esto seria encontrar dos vertices u , v adyacentes y 
    dos conjuntos independientes \textit{head} y \textit{tail } 
    de cardinalidad h y t respectivamente , u ni v pueden estar 
    incluidos en los conjuntos \textit{tail} ni en el conjunto \textit{head} 
    tal que todo los elementos del conjunto \textit{head} sean 
    adyacentes a u y todos los  elementos del conjunto \textit{tail } sean adyacentes a v 

     

    \hspace*{0.5cm} 

    \begin{flushleft}
        Voy a recorrer todas las aristas del grafo G y voy a analizar si 
        esta puede ser la arista que conforma el "cuerpo del hydra"  vamos a 
        denotarla arista e y a los extremos de la arista (los v\'ertices que une) 
        los vamos a llamar u , v para esto vamos a usar la notaci\'on  $u - v$
    \end{flushleft}

    \textbf{Vamos a dividr el analisis de la arista $u-v$}

    \paragraph*{Caso 1 } Si $deg (u) <  h +1$   o el $deg(v) < t+1 $ entonces no hay forma posible 
    de que esta arista forme parte del "cuerpo de hydra "
    
    \paragraph*{Caso 2 } Aqui si tenemos en cuenta que el {\bf Caso 1} no se cumple es porque 
    $deg (u) >= h+1 $ y el $deg(v) >= t+1$  . Este caso lo vamos a partir en 2 . el primer subcaso seria que 
    $deg (u) >= h+t+1$ , aqui si tenemos tambien la condicionante de que $deg (v) >= t+1$ (resultado de que el {\bf caso 1} fuera falso  ) Entonces es posible 
    encontrar los subconjuntos independientes \textit{head}  y \textit{tail } de cardinalidad $h$ y $t$ respectivamente , pues en el peor de los escenarios  que es que 
    todos los adyacentes de v tambien sean adyacentes de u , es posible particionar de forma tal que los subconjuntos se puedan escoger de forma independiente , es segundo 
    subcaso seria $deg(v) > h+t+1$ , que se puede tomar la misma idea del otro subcaso 

    \paragraph*{Caso 3} Teniendo en cuenta que ninguno de los casos anteriores se cumplio es 
    porque se cumple que $h+1 \leq deg(u) \leq h + t + 1 $ y que $t+1 \leq deg(v) \leq h + t + 1 $
    luego habria que determinar aqui si existe o no la estructura que estamos buscando 
    \\*
    Aqui en este caso podemos podar la no existencia con una pregunta :
    Si definimos a U como el conjunto de vetices que son adyacentes a u y a v sin incluir a estos dos 
    entonces podemos saber cuantos de estos v\'ertices que pertenecen a U van a ser necesarios a la hora de 
    conformar los conjuntos head y tail y si esta cantidad es mayor que la cardinalidad del conjunto U entonces no es posible 
    encontrar la estructura que queremos 
    
    \usemintedstyle{solarized-dark}
    \inputminted[bgcolor = bg]{python} {../hydra.py}
    
\end{document}