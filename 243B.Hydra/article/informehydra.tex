\documentclass{article}

\usepackage[spanish,es-noshorthands]{babel}
\usepackage{tikz}
\usetikzlibrary{positioning,arrows.meta}
\usetikzlibrary{shapes,snakes}



\begin{document}
    \title{B.Hydra} 
    \author{Daniel de la Cruz Prieto} 
    \date{\today} 
    \maketitle

    \begin{flushleft}
    	Un d\'ia Petya recibi\'o un regalo de cumplea\~nos de su mam\'a, un libro llamado "Leyes y mitos de la Teor\'ia de Grafos". De este libro Petya aprendi\'o acerca de los grafos Hydra.
    \end{flushleft}
    
    \begin{flushleft}
        Un grafo no dirigido es un Hydra; si este tiene una estructura como la mostrada en la figura de abajo.\\
        Esta estructura tiene dos nodos $u$ y $v$ conectados por una arista. Estos son el pecho y el estomago de Hydra
        correspondientemente. El pecho es conectado con h nodos que son la cabeza de Hydra. El est\'omago est\'a conectado con t nodes que son la cola de Hydra.
    \end{flushleft}
      
    \begin{flushleft}
         Note que el Hydra es un \'arbol, que tiene $h+t+2$ nodos
    \end{flushleft}
    
    \begin{center}
        \begin{tikzpicture}[vertex/.style={draw,circle} ]
        \tikzstyle{stylenode} = [vertex ,fill = red ,draw = none ,text = black ,scale = 1]
            
            \node[ stylenode] (1) at (1.6,6) {} ; 
            \node[ stylenode] (2) at (1.4,5) {} ; 
            \node[ stylenode] (3) at (1.3,4) {} ; 
            \node[ stylenode] (4) at (3,5) {$u$} ; 

            \node[ stylenode] (5) at (7,4) {$v$} ; 
            \node[ stylenode] (6) at (9,5) {} ; 
            \node[ stylenode] (7) at (9,4) {} ; 
            \node[ stylenode] (8) at (9,3) {} ; 
            \node[ stylenode] (9) at (8.4,2) {} ; 
            


            \draw [line width = 2] (1) -- (4);
            \draw [line width = 2] (2) -- (4);
            \draw [line width = 2] (3) -- (4);

            \draw [line width = 2] (4) -- (5);

            \draw [line width = 2] (5) -- (6);
            \draw [line width = 2] (5) -- (7);
            \draw [line width = 2] (5) -- (8);
            \draw [line width = 2] (5) -- (9);

            \node[right = 0.5cm,text width = 4cm,font=\footnotesize,scale = 1.5] at (1)
            {
                heads
            };
            \node[right = 1.5cm,text width = 4cm,font=\footnotesize,scale = 1.5] at (4)
            {
                body
            };
            \node[left = 0.5cm,text width = 0.5cm,font=\footnotesize,scale = 1.5] at (6)
            {
                tails
            };

        \end{tikzpicture}
    \end{center}

    \begin{flushleft}
        Tambien, Petya tiene un grafo no dirigido G que consiste de n nodos y m aristas. Petya recibi\'o este como el \'ultimo regalo de cumplea\~nos de su mam\'a. El grafo G no contiene ni multiples aristas ni lazos.
    \end{flushleft}
  
    \begin{flushleft}
        Ahora Petya quiere encontrar un Hydra en el grafo G o estar segura que el grafo no tiene un Hydra.
    \end{flushleft}
\end{document}