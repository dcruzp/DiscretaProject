\documentclass[12pt]{article}

\usepackage[spanish,es-noshorthands]{babel}
\usepackage{tikz}
\usepackage{minted}
\usepackage[hidelinks]{hyperref} 
\usepackage{color}
\usepackage{amsmath}
\usepackage{algpseudocode}
\usepackage{algorithm}
\usepackage{makeidx} 

\definecolor {bg} {rgb}{0.95,0.95,0.95}


\usetikzlibrary{positioning,arrows.meta}
\usetikzlibrary{shapes,snakes}

\renewcommand{\algorithmicrequire}{\textbf{Input:}}
\renewcommand{\algorithmicensure}{\textbf{Output:}}


\begin{document}
    \title{B.Hydra} 
    \author{Daniel de la Cruz Prieto} 
    \date{\today} 
    \maketitle

    \begin{abstract}
        \noindent En este art\'iculo se expone la soluci\'on del problema B.Hydra del juez online \href{https://codeforces.com/} { \textcolor{blue}{CodeForce} }  
        si quiere ir directo al link del problema \href{https://codeforces.com/problemset/problem/243/B}{\textcolor{blue}{click aqui}} . En el art\'iculo se da una soluci\'on al problema , adem\'as de 
        demostrar la correctitud del algoritmo y la complejidad temporal del mismo . Al final del articulo se expone el c\'odigo en python de la soluci\'on del problema   
    \end{abstract}

    
    \section*{Orientaci\'on del Problema }
    \begin{flushleft}
        Un d\'ia Petya recibi\'o un regalo de cumplea\~nos de su mam\'a, un libro llamado "Leyes y mitos de la Teor\'ia de Grafos". De este libro Petya aprendi\'o acerca de los grafos Hydra.
        \\*
        Un grafo no dirigido es un Hydra; si este tiene una estructura como la mostrada en la figura de abajo.
        \\*
        Esta estructura tiene dos nodos $u$ y $v$ conectados por una arista. Estos son el pecho y el estomago de Hydra
        correspondientemente. El pecho es conectado con h nodos que son la cabeza de Hydra. El est\'omago est\'a conectado con t nodes que son la cola de Hydra.
        \\*
        Note que el Hydra es un \'arbol, que tiene $h+t+2$ nodos
    \end{flushleft}
    
    \begin{center}
        \begin{tikzpicture}[vertex/.style={draw,circle} ]
        \tikzstyle{stylenode} = [vertex ,fill = red ,draw = none ,text = black ,scale = 1]
            
            \node[ stylenode] (1) at (1.6,6) {} ; 
            \node[ stylenode] (2) at (1.4,5) {} ; 
            \node[ stylenode] (3) at (1.3,4) {} ; 
            \node[ stylenode] (4) at (3,5) {$u$} ; 

            \node[ stylenode] (5) at (7,4) {$v$} ; 
            \node[ stylenode] (6) at (9,5) {} ; 
            \node[ stylenode] (7) at (9,4) {} ; 
            \node[ stylenode] (8) at (9,3) {} ; 
            \node[ stylenode] (9) at (8.4,2) {} ; 
            


            \draw [line width = 2] (1) -- (4);
            \draw [line width = 2] (2) -- (4);
            \draw [line width = 2] (3) -- (4);

            \draw [line width = 2] (4) -- (5);

            \draw [line width = 2] (5) -- (6);
            \draw [line width = 2] (5) -- (7);
            \draw [line width = 2] (5) -- (8);
            \draw [line width = 2] (5) -- (9);

            \node[right = 0.5cm,text width = 4cm,font=\footnotesize,scale = 1.5] at (1)
            {
                heads
            };
            \node[right = 1.5cm,text width = 4cm,font=\footnotesize,scale = 1.5] at (4)
            {
                body
            };
            \node[left = 0.5cm,text width = 0.5cm,font=\footnotesize,scale = 1.5] at (6)
            {
                tails
            };

        \end{tikzpicture}
    \end{center}

    \begin{flushleft}
        Tambien, Petya tiene un grafo no dirigido G que consiste de n nodos y m aristas. Petya recibi\'o este como el \'ultimo regalo de cumplea\~nos de su mam\'a. El grafo G no contiene ni multiples aristas ni lazos.
    \end{flushleft}
  
    \begin{flushleft}
        Ahora Petya quiere encontrar un Hydra en el grafo G o estar segura que el grafo no tiene un Hydra.
    \end{flushleft}


    \vspace{1cm}
    \noindent \textcolor{black}{\Large Entrada }
    \begin{flushleft}
        La primera linea contine cuatro enteros n,m,h,t $\left(1\leq n,m\leq {10}^{5} , 1 \leq h , t,\leq 100\right)$, el n\'umero de nodos y las 
        aristas en el grafo \textit{G} y el n\'umero de la cabeza del Hydra y la cola.
        \\*
        Las proximas m lineas continen la descricion de las aristas del grafo \textit{G} . La $i-esima$ de esas lineas 
        continen dos enteros $a_i$ y $b_i$ , $\left(1 \leq a_i , b_i \leq n , a \neq b\right)$ el n\'umero de los nodos conectados por la $i-esima$ arista. 
        \\*
        Se garantiza que el grafo \textit{G} no contine lazos ni aristas multiples . Considera los nodos del grafo \textit{G} numerados desde 1 hasta n.  
    \end{flushleft}

    \vspace{0.5cm}
    \noindent \textcolor{black}{\Large Salida }
    \begin{flushleft}
        Si el grafo \textit{G} no contiene un hydra entonces imprimir -NO-.
        \\*
        En otro caso , en la primera linea imprimir -YES- . En la segunda linea imprimir dos enteros  - los n\'umeros de los nodos u y v. 
        En la tercera linea imprimir h n\'umeros  - los n\'umeros de los nodos que son la cabeza. En la cuarta linea imprimir t n\'umeros  - los n\'umeros de los nodos que son los pies. 
        Todos los n\'umeros que se impriman deben de ser distintos.
        \\*
        Si hay m\'ultiples posibles respuestas , tienes permitido imprimir cualquiera de ellas.   
    \end{flushleft}

    \noindent \textbf{\Large Ejemplo 1}
    \begin{minted}[bgcolor = bg , frame = single , framerule = 1pt , framesep = 5pt , gobble = 8 , label = INPUT] {console}
        9 12 2 3
        1 2
        2 3
        1 3
        1 4
        2 5
        4 5
        4 6
        6 5
        6 7
        7 5
        8 7
        9 1
    \end{minted}

    \begin{minted}[bgcolor = bg , frame = single , framerule = 1pt , framesep = 5pt , gobble = 8 , label = OUTPUT] {console}
        YES
        4 1
        5 6 
        9 3 2
    \end{minted}
    
    \vspace{1cm}
    \noindent \textbf{\Large Ejemplo 2}
    \begin{minted}[bgcolor = bg , frame = single , framerule = 1pt , framesep = 5pt , gobble = 8 , label = INPUT] {console}
        7 10 3 3
        1 2
        2 3
        1 3
        1 4
        2 5
        4 5
        4 6
        6 5
        6 7
        7 5
    \end{minted}
    
    \begin{minted}[bgcolor = bg , frame = single , framerule = 1pt , framesep = 5pt , gobble = 8 , label = OUTPUT] {console}
        NO
    \end{minted}

    \begin{algorithm}[h]
        \caption{Calculo de \textit{first hop}}
        \begin{algorithmic}[1]
       
          % ENTRADA / SALIDA
          \Require{ G , u , v } 
          \Ensure{head , tail}
       
          \State countcommun = 0 
          \ForAll {$a \in Ady \left[v\right] $} sjdfg\EndFor
       
          \For{$i;\ i++;\ |target|$}
            \State{traceroute($IP = IP[i],\ TTL_{initial} = 1$)}
          \EndFor
       
          \For{$i;\ i++;\ |target|$}
            \State{trace = read\_traceroute($IP[i]$)}
            \State{$TTL_{initial} = 1$}
            \For{hop \textbf{in} trace}
              \If{hop \textbf{in} $IP_{found}$}
                \State $TTL_{initial} ++$
              \Else
                 \State $IP_{found}$ \textbf{append} hop
                 \State $first\_hop[i] = TTL_{initial}$
                 \State \textbf{break}
              \EndIf
            \EndFor
          \EndFor
          \State{\textbf{return} $first\_hop$}
        \end{algorithmic}
      \end{algorithm}

    
    \section{An\'alisis del problema }

    \subsection{Dicuci\'on de lo que nos piden en el problema } 
    \noindent Tenemos que dado un grafo G no dirigido y dos entros h y t determinar si 
    existe alguna estuctura de la forma descrita en la orden del ejercicio. 
    Esto ser\'ia encontrar dos v\'ertices u , v adyacentes y 
    dos conjuntos independientes \textit{head} y \textit{tail } 
    de cardinalidad h y t respectivamente ,donde ni el v\'ertice u ni el v\'ertice v pueden estar 
    incluidos en los conjuntos \textit{tail} ni en el conjunto \textit{head}. 
    Tambi\'en se tiene que cumplir que todo los v\'ertice  del conjunto \textit{head} sean 
    adyacentes a u y todos los  v\'ertices del conjunto \textit{tail } sean adyacentes a v.

     

    \subsection*{Ideas para la resoluci\'on del problema }
    
    \noindent La primera idea que se me ocurri\'o para hacer esto fue recorrer todas 
    las aristas del grafo y ver si podia conformar un Hydra donde la arista que estaba 
    analizando era el Hydra del Grafo .
    \\*
    Luego de pensar el ejercicio de esta forma empece a analizar de que forma pod\'ia armar un Hydra 
    en el un grafo dado con la idea anterior  .

    \section{Soluci\'on del Problema }

    \noindent Voy a recorrer todas las aristas del grafo G y voy a analizar si 
    esta puede ser la arista que conforma el "cuerpo del hydra"  vamos a 
    denotarla arista e y a los extremos de la arista (los v\'ertices que une) 
    los vamos a llamar u , v para esto vamos a usar la notaci\'on  $u \leftrightarrow v$.
    Entonces el an\'alisis de una arista $e$ lo vamos  a hacer de la siguiente manera. 
    

    \subsection{An\'alisis de la arista e}

    \noindent Vamos a analizar la arista $e$ de forma tal que vamos a asumir que para nuestro an\'alisis la arista $e$ 
    es tomada en el siguiente sentido $u\xrightarrow[]{e}v$ y analizar si el extremo $u$ de la arista puede formar 
    la "cabeza" del Hydra (o sea que los v\'ertices adyacentes a $u$ que estoy buscando sean los que conformen el conjunto 
    \textit{head}) y el extremo $v$ sea el adyacente a los v\'ertices que conforman el \textit{tail} . Como el grafo es no 
    dirigido hay que analizar tambien el otro sentido de la arista , es decir considerar a $e$ de la forma $v\xrightarrow[]{e}u$ 

    \noindent Vamos a ir descartando el an\'alisis de algunas aristas y tratar de buscar la posible o no soluci\'on
    en otras que cumplan ciertos requisistos y para esto  voy a dividir el an\'alisis por casos

    \subsubsection{An\'alisis de casos } 

    \begin{itemize}
        \item \textit{\textbf {Caso 1 }} Si $deg (u) <  h +1$   o el $deg(v) < t+1 $ entonces no hay forma posible 
        de que esta arista forme parte del "cuerpo de hydra " . Es decir no puedo encontrar los conjuntos \textit{head} y 
        \textit{tail} de carinalidad h y t respectivamente porque $u$ y $v$ no tienen las los adyacentes necesarios para armar estos 
        conjuntos con las especificaciones que discutimos arriba 

        \item  \textit{\textbf {Caso 2}} Aqu\'i si llegamos a este an\'alisis es porque suponemos que el caso uno fue analizado. 
        Luego si el  \textit{Caso 1} no se cumple es porque las condiciones $deg (u) \geq h+1 $ y  $deg(v) \geq t+1$ se cumplen ambas a la vez.
        Teniendo en cuenta esto vamos ahora a analizar lo siguiente: 
        Si suponemos ahora que $deg (u) \geq h+t+1$  Entonces es posible  encontrar los subconjuntos independientes \textit{head}  y \textit{tail } de cardinalidad $h$ y $t$ respectivamente que
        nos piden en el ejercicio , pues en el peor de los escenarios  que es que todos los adyacentes de $v$ tambi\'en sean adyacentes de $u$ pero $u$ tiene suficientes adyacentes para particionar
        de forma tal que sea posible escoger los subconjuntos con las restricciones que analizamos arriba
        \\*
        En el caso en que no se cumpla que $deg (u) \geq h+t+1$ podemos analizar si $deg(v) \geq h+t+1$ , donde se puede formar de igual manera los conjuntos que andamos buscando 
        
        \item  \textit{\textbf{Caso 3}} Teniendo en cuenta que n\'inguno de los casos anteriores se cumpli\'o es 
        porque se cumple que $h+1 \leq deg(u) \leq h + t  $ y adem\'as se cumple que $t+1 \leq deg(v) \leq h + t  $
        luego habr\'ia que determinar aqui si existe o no la estructura que estamos buscando 
        \\*
        En este caso podemos comprobar la no existencia de un hidra de la siguiente manera :
        Si definimos a $U$ como el conjunto de v\'etices que son adyacentes a $u$ y a $v$ sin incluir a estos dos (porque si estamos analizando
        esto es porque esiste una arista que conecta a $u$ con $v$ ) , y en las restricciones del problema nos dicen 
        que $u$ y $v$ no pueden pertenecer a los conjunto  que estamos buscando, entonces podemos saber cu\'antos de estos v\'ertices
        que pertenecen a $U$ van a ser necesarios a la hora de conformar los conjuntos \textit{head} y \textit{tail} y si esta cantidad 
        es mayor que la cardinalidad del conjunto $U$ entonces no es posible  encontrar la estructura que queremos. 
    \end{itemize}


    \subsection{An\'alisis detallado del Caso 2}

    \noindent ?` La pregunta aqu\'i ser\'ia como formar los conjuntos que nos piden si se cumple que $deg (u) \geq h+t+1$  y que $deg(v) \geq t+1$ ? 
    La idea es recorrer todos los adyacentes de $v$ y escoger $t$ v\'ertices de estos para armar el conjunto \textit{tail} y marcarlos para saber
    que ya forman parte del conjunto \textit{tail}  , despues recorro los adyacentes de $u$ y escojo t de estos que no se hayan escogidos para el
    conjunto \textit{tail} , esto se puede hacer perfectamente pues $u$ tiene suficientes adyacentes para escoger y armar el \textit{head} una vez sea
    hayan escogidos los v\'ertices del \textit{tail} . 
    \\*
    Es importante aqu\'i hacerlo en el orden en que lo describ\'i (primero el \textit{tail} y despu\'es el \textit{head}), pues  si se hace en el otro orden 
    puede ocurrir que todos los que escog\'i para conformar el \textit{head} sean todos los comunes  o gran parte de ellos  (adyacentes a $u$ y a $v$ a la vez), 
    y despues cuando vaya aconformar el \textit{tail} no pueda hacerlo porque me faltan por escoger y los restantes ya hayan sido escogidos para conformar el \textit{head}. 
    Sin embargo esto no ocurre si primero escojo los vertices del \textit{tail } y despues los v\'ertices del \textit{head}
    \\*
    El mismo procedimiento se puede hacer para el caso en que $deg (v) \geq h+t+1$ y $deg(u) \geq h+1$ . Siempre tenemos que tener en cuenta primero escoger 
    (en este caso ) primero el \textit{head} y despues el \textit{tail} 
    

    \subsection{An\'alisis detallado del caso 3}

    \noindent Si Llegamos al caso 3 tenemos que se cumple que  $h+1 \leq deg(u) \leq h + t  $ y adem\'as se cumple que $t+1 \leq deg(v) \leq h + t  $
    \\*
    ?`Como saber si existe un Hydra en este caso o no ? 
    \\*
    Primero vamos a analizar un caso en el que no puede existir un Hydra . Y la idea es la siguiente: 
    Vamos a denotar a $C$ como el conjunto formado por los  adyacentes a $u$ que son tambi\'en adyacentes de $v$ .
    \\* 
    Vamos a denotar a $IndU$ como el conjunto de adyacentes de $u$ que no est\'an en $C$ , ($v \notin IndU$) 
    \\*
    Y vamos a denotar a $IndV$ como el conjunto de adyacente de $v$ que no est\'an en $C$ , ($u \notin IndV$)
    \\*
    Si se cumple que $ h -\vert IndU \vert  +  t - \vert IndV \vert > \vert C \vert $ entonces no se puede armar un Hydra . Porque la cantidad de v\'ertices necesarios 
    que tengo que escoger que son comunes para formar el \textit{tail } y el \textit{head} es mayor que la cantidad de v\'ertices que estos comparten en com\'un. 
    \\[5pt] 
    En caso que lo que analizamos antes no se cumpla entonces si podemos escoger el \textit{head} y el \textit{tail} 
    \\*
    ?` como hacemos esto ? 
    \\*
    Si grarantizamos que a la hora de escoger el conjunto \textit{tail} no escogemos una cantidad de  v\'ertices superior a los que son necesarios escoger del conjunto $C$ para formar el \textit{tail} entonces 
    estamos garantizando que vamos a poder despu\'es escoger el \textit{head} sin ning\'un impedimento , Igual pasa si escogemos primero los v\'ertices que conforman el conjunto \textit{head} 

    \subsection{Seudoc\'odigo del algor\'itmo} 

    \noindent Vamos a anlizar el seudoc\'odigo por partes : 

    \begin{itemize}
        \item \textbf{Caso 1} 
        \begin{algorithmic}
            \If {$deg\left(u\right) < h+1$ \textbf{or} $deg\left(v\right) < t+1$}  
                \State \textbf{return} 
            \EndIf 
        \end{algorithmic}
        No es posible la existencia de un Hydra 

        \item \textbf{Caso 2}
        
        \begin{algorithmic}
            \If {$deg\left(u\right) \geq h+t+1$}
                \ForAll {$a \in Ady \left[v\right]$}
                    \If{$tail.size == t $}
                        \State \textbf{break} 
                    \EndIf
                    \If{$a==u$}
                        \State \textit{continue}
                    \EndIf 
                    \State $a \leftarrow $ visited 
                    \State append (tail , a ) 
                \EndFor 
                \ForAll {$a \in Ady \left[u\right]$}
                    \If{$head.size == h $} 
                        \State\textbf{break}
                    \EndIf
                    \If{$a==u$} 
                        \State \textit{continue} 
                    \EndIf 
                    \State $a \leftarrow $ visited 
                    \State append (head , a ) 
                \EndFor 
                \State \textbf{RETURN}
            \EndIf
        \end{algorithmic}

        En este subcaso del caso 2 vimos que era posible armar un Hydra y el procedimiento es el descrito en el seudoc\'odigo de arriba 
        \\*
        Ahora para el otro subcaso pasa igual solo que debemos escoger primero el \textit{head} y despu\'es el \textit{tail} , el procedimineto 
        es el descrito en el seudocodigo de abajo  
        \vspace{0.5cm}
        \begin{algorithmic}
            \If {$deg\left(v\right) \geq h+t+1$}
                \ForAll {$a \in Ady \left[u\right]$}
                    \If{$head.size == h $}
                        \State \textbf{break} 
                    \EndIf
                    \If{$a==v$}
                        \State \textit{continue}
                    \EndIf 
                    \State $a \leftarrow $ visited 
                    \State append (head , a ) 
                \EndFor 
                \ForAll {$a \in Ady \left[v\right]$}
                    \If{$tail.size == t $}
                        \State \textbf{break} 
                    \EndIf
                    \If{$a==u$ \textbf {or} \textit{a is visited}}
                        \State \textit{continue}
                    \EndIf 
                    \State append (tail , a ) 
                \EndFor 
                \State \textbf{RETURN}
            \EndIf 
        \end{algorithmic}

        \item \textbf{Caso 3}
        
        Para este caso primero vamos a encontrar los v\'ertices adyacentes que tienen en com\'un $u$ y $v$ y vamos a determinar  $\vert Indu \vert $  y $\vert Indv \vert $ para seber si 
        es posible armar un Hydra o no 
        \vspace{0.3cm}
        \begin{algorithmic}
            \State \textit{commun} $\leftarrow$ $vertexcommun \left(u,v\right)$ 
            \State \textit{Indu} $\leftarrow$ $deg\left(u\right) -1 - commun$
            \State \textit{Indv} $\leftarrow$ $deg\left(v\right) -1 - commun$
            \If {$h - Indu + t - Indv  > commun$}
                \State \textit{$restorevisited \left(u\right)$}
                \State \textbf{Return}
            \EndIf 
        \end{algorithmic}

        Para encontrar los v\'ertices en com\'un (m\'etodo \textit{vertcommun()})lo hacemos de la siguiente manera:  

        \begin{algorithmic}
            \State $commun \leftarrow  0$ 
            \ForAll{$a \in Ady \left[u\right]$}
                \If {$a = v$}
                    \State \textbf{continue}
                \EndIf
                \State a $\leftarrow$ visited 
            \EndFor
            \ForAll{$a \in Ady \left[v\right]$}
                \If {\textit{a is visited }}
                    \State $commun = commun +1 $
                \EndIf
            \EndFor
            \State \textbf{return} \textit{commun} 
        \end{algorithmic}

        Despu\'es de la llamada al m\'etodo  \textit{vertcommun()} todos los adyacentes $u$ excepto $v$ quedaron visitados
        \\*
        Si se cumple la condici\'on   $h - Indu + t - Indv  > commun$ es porque no se puede formar un Hydra , y el metodo en grneral debe terminar aqui , 
        pero el array \textit{visited} tiene a todos los adyacentes de $u$ marcados como visistados , por lo que necesito revertir esto , y para ello llamo 
        al m\'etodo \textit{restorevisited ()} cuyo seudoc\'odigo es el de abajo:   
        
        \vspace{0.5cm}

        \begin{algorithmic}
            \ForAll{$a \in Ady \left[u\right]$}
                \State $a$ $\leftarrow$ $not$ $visited$
            \EndFor
        \end{algorithmic}

        Entonces el array qued\'o "l\'impio" para poder analizar otra arista 
        
        \vspace{0.3cm} 

        En el caso en que $h - Indu + t - Indv  \leq commun$ es posible armar un Hydra y en el seudocodigo de abajo describe el procedimiento para encontrarlo
        
        \begin{algorithmic}
            \State $count = 0 $ 
            \ForAll {$a \in Ady\left[v\right]$}
                \If {$tail.size == t$}
                    \State \textit{break}
                \EndIf
                \If {$a$ is visited \textbf{and} count $<$ $t - Indv $}
                    \State count = count + 1 
                    \State append $\left(tail , a \right)$
                \Else 
                    \If{a is not visited \textbf{and} a $\neq$ u}
                        \State append $\left(tail , a \right)$
                    \EndIf
                \EndIf 
            \EndFor 
            \ForAll{a $\in$ $Ady\left[u\right]$}
                \If {$head.size == h$}
                    \State \textbf{break}
                \EndIf
                \If{a is not visited }
                    \State continue 
                \EndIf
                \State $append \left(head ,a \right) $
            \EndFor
            \State \textbf{Return}
        \end{algorithmic}
    \end{itemize}


    \section{Correctitud del algoritmo }
    \noindent EL algoritmo termina pues una vez analizadas todas las arista el algoritmo finaliza , (en caso de encontrar un Hydra )este termina 
    instanatniamente el analisis de las aristas restantes 
    \\*
    Lo mas importante es ver que todos los posibles escenarios estan bien detallados arriba , luego para cada situaci\'on el algoritmo termina dando la 
    salida esperada .
    \\*
    El primer caso es trivial . Lo que habr\'ia que analizar bien  si en el segundo caso y en el tercero el algoritmo se comporta como esperamos . En el segundo caso es sencillo 
    ver que  si armamos los conjuntos que queremos dar como salida en el orden en que se describe no puede haber contradiccion . Igual pasa con el tercer caso . 
    \\*
    Como es un algoritmo constructivo es facil ver todas los posibles escenarios y para cada una de ellos establecer un comportamineto .


    \section{Complejidad Temporal }

    \noindent El algoritmo recorre todas las aristas del grafo , como es no dirigido se analizan en los dos sentidos ,  pero al final su complejidad temporal es $O \left(\vert E \vert\right)$.
    \\*
    Vamos a analizar la complejidad temporal que requiere aplicar nuestro algoritmo descrito a una arista en espec\'ifico y lo vamos a hacer por cada uno de los casos en que se divide el algoritmo
    \\*
    Para el primer caso su complejidad es $O\left(1\right)$
    \\*
    Para el segundo caso como lo que hace el algoritmo es recorrer los adyacentes de  u y de v y armar los conjuntos \textit{head} y \textit{tail} esto tiene una complejidad de  $O\left(h+t\right)$ . esto se puede ver pues en cada uno de los recorridos por los adyacentes 
    de u y de v cuando termina de escoger los conjuntos , el recorrido termina , pues el proceso de inserci\'on en los conjuntos no se realiza mas de $h+t$ veces
    \\*
    En el tercer caso primero analizamos los adyacentes de $u$ y de $v$ para saber cuales son v\'ertices en comun , pero una de las condiciones por las que el algoritmo entra en el caso 3 es porque se cumpl\'ia que $deg\left(u\right) \leq h+t$ y el $deg\left(v\right) \leq h+t$ si no hubiese ca\'ido en el segundo caso , 
    Luego el procedimiento para saber la cantidad de v\'ertices en  com\'un tiene una complejidad de $O\left(h+t\right)$  
    \\*
    En caso de no ser posible encontrar un Hydra entonces revertir los v\'ertices que fueron masrcados como vistitados tiene una complejidad temporal de $O\left(h+t\right)$
    \\*
    Y si es posible armar el hydra entonces tenemos la misma complejidad temporal que el caso 2 . 
    \\*
    Por lo que la complejidad temporal del algoritmo en general es $O\left(\vert E \vert \cdot \left(h+t\right)\right)$ 

    \section{Mi implementaci\'on}

    \noindent Mi implementaci\'on la hice primero en c++ y despu\'es en python  . El c\'odigo  que probe en el CodeForce fue el de c++ .El c\'odigo de python lejos de estar mal porque es una traduccion exacta del codigo de c++ , pude que se vaya de tiempo 

    \subsection{C\'odigo en c++ }

    \noindent EL metodo \textit{solution()} es el que recorre a todas las aristas y devuelve la soluci\'on al problema , es c\'odigo es el de abajo: 
    \begin{minted}[gobble = 8 , breaklines ,bgcolor = bg,frame = single ,linenos = true , label = solution ]{cpp} 
         void solution ()
        {
            int i , u , v ; 
        
            for (i=0 ; i < aristas.size(); i++) 
            {
                u = aristas[i].first ;
                v = aristas[i].second;

                if (checkEdge(u,v)) 
                {
                    printHydra(u,v);
                    return ; 
                }
                else if (checkEdge(v,u)) 
                {
                    printHydra(v,u);
                    return ;
                }
            }
            cout << "NO" <<endl ; 
        }       
    \end{minted}

    \noindent El m\'etodo \textit{chackEdge()} es el que analiza una arista para saber si puede se puede tomar a esta como el cuepo del hydra y formar los conjuntos 
    \textit{head} , \textit{tail} , en caso de que no se pueda el array visistes los devuelve "l\'impio" (o sea todos los vertices quedan como no visitados ), el  c\'odigo es el de abajo

    \begin{minted}[gobble = 8 , breaklines ,bgcolor = bg,frame = single ,linenos = true , label = checkEdge ]{cpp}        
        int checkEdge (int u , int v ) 
        {
            int deg_u = graph[u].size();   
            int deg_v = graph[v].size();    
            int count , i , a , ind_u , ind_v , commun; 
          
            if (deg_u < h+1 || deg_v < t+1)
            {
                return 0 ; 
            }

            if (deg_u >= h+t+1)
            {
                buildSets1(u,v); 
                return 1 ;         
            }
            if (deg_v >= h+t+1)
            {
                buildSets2(u,v);   
                return 1 ;        
            }
        
            commun = vertexcomun(u,v);  
            ind_u = deg_u -1 -commun ;  
            ind_v = deg_v -1 -commun ;  
        
            if ( (h- ind_u + t - ind_v ) > commun) 
            {
                restorevisitedarray(u);  
                return 0 ;      
            }

            buildSets3(u,v,commun); 
            return 1 ;  
        }

    \end{minted}

    \noindent el m\'etodo \textit{ buildSets1()}  \textit{ buildSets2()} \textit{ buildSets3()} son los  que se encargan de construir los conjuntos \textit{head} , \textit{tail} teniendo en consideraci\'on que es lo que hay que tener en cuenta para construir los conjuntos en dependencia de que caso estamos analizando 
    \\*
    Los c\'odigos los presento a continuaci\'on 
    
    \begin{minted}[gobble = 8 , breaklines ,bgcolor = bg,frame = single ,linenos = true , label = buildSets1 ]{cpp}
        void buildSets1 (int u , int v)
        {
            int deg_u = graph[u].size(); 
            int deg_v = graph[v].size(); 
            int i , a ; 
            for (i=0 ; i < deg_v ; i++)
            {
                if (tail.size() == t ) break;  
                a = graph[v][i]; 
                if (a == u ) continue ;     
                visited[a] = true ;  
                tail.push_back (a) ;   
            }
            for (i=0 ; i < deg_u ; i++)
            {
                if (head.size() == h )break; 
                a = graph[u][i] ; 
                if (visited[a] || a == v ) 
                {
                    continue ; 
                }
                head.push_back(a);  
            }
        }
    \end{minted}

    \begin{minted}[gobble = 8 , breaklines ,bgcolor = bg,frame = single ,linenos = true , label = buildSets2 ]{cpp}
        void buildSets2 (int u , int v)
        {
            int deg_u = graph[u].size(); 
            int deg_v = graph[v].size(); 
            int i , a ; 
            for (i=0 ; i < deg_u ; i++)
            {
                if (head.size() ==h)break; 
                a = graph[u][i]; 
                if (a == v ) continue ; 
                visited[a] = true ; 
                head.push_back (a) ; 
            }
            for (i=0 ; i < deg_v ; i++)
            {
                if (tail.size() == t) break ;
                a = graph[v][i] ; 
                if (visited[a] || a == u ) continue ; 
                tail.push_back(a); 
            }
        }
    \end{minted}

    \begin{minted}[gobble = 8 , breaklines ,bgcolor = bg,frame = single ,linenos = true , label = buildSets3 ]{cpp}
        void buildSets3 (int u , int v ,int commun ) 
        {
            int deg_u = graph[u].size();   
            int deg_v = graph[v].size();   
            int i , a ; 
            int count = 0 ; 
            for (i=0 ; i<deg_v ; i++)
            {
                if (tail.size() == t ) break; 
                a = graph[v][i]; 
                if (visited[a] && count < t - (deg_v-1 -commun) ) 
                {
                    count = count + 1 ; 
                    visited[a] = false ; 
                    tail.push_back(a);
                }
                else if (!visited[a] && a != u ) 
                {
                    tail.push_back(a);
                }
            }
            for (i=0 ; i<deg_u ; i++)
            {
                if (head.size() == h) break; 
                a = graph[u][i];
                if (!visited[a]) continue;  
                head.push_back(a);
            }
        }
    \end{minted}

    

    \noindent El c\'odigo en c++ presentado arriba esta en esta misma carpeta donde se encuentra el proyecto , el fichero se llamas \textit{hydra.cpp} , la complejidad temporal es la misma descrita arriba , las bibliotecas que usa son las estandares de c++ y por lo tanto la inserci\'on  y consulta del tamano de los array es $O\left(1\right)$
    
    \subsection{C\'odigo en Python} 

    \noindent M\'etodo \textit{solution ()} 
    
    \begin{minted}[gobble = 8 , breaklines ,bgcolor = bg,frame = single ,linenos = true , label = solution ]{python}
        def solution ():
            for item in edges :  # recorre todas las aristas 
                u , v = item[0] , item [1]

                if checkEdge(u,v):
                    printsolution(u,v)
                    return 
                elif checkEdge (v,u):
                    printsolution(v,u)
                    return 
            print("NO")
    \end{minted}

    \begin{minted}[gobble = 8 , breaklines ,bgcolor = bg,frame = single ,linenos = true , label = checkEdge ]{python}
        def checkEdge (u,v):
            deg_u = len (graph[u])  # degree del vertice u 
            deg_v = len (graph[v])  # degree del vertice v
            
            #Caso 1 
            if deg_u < h+1 or deg_v < t+1:
                return False 

            #Caso 2
            if deg_u >= h+t+1:
                buildSet1(u,v)
                return True
            if deg_v > h+t+1:
                buildSet2(u,v)
                return True 
            
            #Caso 3 
            commun = vertexcommun(u,v) 
            ind_u = deg_u -1 - commun  
            ind_v = deg_v -1 - commun   

            if h - ind_u + t - ind_v > commun:  
                restorevisitedarray(u)
                return False 

            buildSet3(u,v ,commun)
            return True
    \end{minted}

    \begin{minted}[gobble = 8 , breaklines ,bgcolor = bg,frame = single ,linenos = true , label = buildSet1 ]{python}
        def buildSet1 (u,v):
        deg_u = len (graph[u]) # degree del vertice u 
        deg_v = len (graph[v]) # degree del vertice v

        #determinar el tail 
        i=0
        while i < deg_v:
            if len(tail) == t : break 
            a = graph[v][i]
            i = i + 1 
            if a == u : continue 
            visited[a] = True 
            tail.append(a)

        #determinar el head
        i=0
        while i < deg_u:
            if len(head) == h : break
            a = graph[u][i]
            i = i +1 
            if a == v  or visited[a]: continue 
            head.append(a)
    \end{minted}

    \begin{minted}[gobble = 8 , breaklines ,bgcolor = bg,frame = single ,linenos = true , label = buildSet2 ]{python}
        def buildSet2 (u,v):
            deg_u = len (graph[u]) 
            deg_v = len (graph[v])
            i=0 
            while i  < deg_u:
                if len(head) == h :break
                a = graph[u][i]
                i = i + 1 
                if a == v : continue 
                visited[a]= True 
                head.append(a)
            i=0 
            while i < deg_v:
                if len(tail) == t : break
                a= graph[v][i]
                i = i+1
                if a == u or visited[a] : continue
                tail.append(a)
    \end{minted}

    \begin{minted}[gobble = 8 , breaklines ,bgcolor = bg,frame = single ,linenos = true , label = buildSet3 ]{python}
        def buildSet3 (u,v , commun):
            deg_u = len (graph[u]) 
            deg_v = len (graph[v]) 

            count = 0 
            i=0
            while i < deg_v:
                if len(tail) == t : break 
                a = graph[v][i]
                i = i + 1
                if visited[a] and count < t - (deg_v-1 -commun):     
                    count = count + 1
                    visited[a] = False 
                    tail.append(a)
                elif visited[a] == False  and a != u :
                    tail.append(a)
            i=0 
            while i < deg_u:
                if len(head)==h : break 
                a = graph[u][i]
                i=i+1
                if  visited[a] == False : continue 
                head.append(a) 
    \end{minted}

    \noindent El c\'odigo de Python descrito arriba se encuentra completo en la carpeta de este informe en el fichero \textit{hydra.py} . en este c\'odigo no se usa ninguna estructura especial por lo que su complejidad temporal es la misma que la descrita arriba 
\end{document}