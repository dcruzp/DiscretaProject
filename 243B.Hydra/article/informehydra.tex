\documentclass[12pt]{article}

\usepackage[spanish,es-noshorthands]{babel}
\usepackage{tikz}
\usepackage{minted}
\usepackage[hidelinks]{hyperref} 
\usepackage{color}
\usepackage{amsmath}
\usepackage{algpseudocode}
\usepackage{algorithm}
\usepackage{makeidx} 

\definecolor {bg} {rgb}{0.95,0.95,0.95}


\usetikzlibrary{positioning,arrows.meta}
\usetikzlibrary{shapes,snakes}

\renewcommand{\algorithmicrequire}{\textbf{Input:}}
\renewcommand{\algorithmicensure}{\textbf{Output:}}


\begin{document}
    \title{B.Hydra} 
    \author{Daniel de la Cruz Prieto} 
    \date{\today} 
    \maketitle

    \begin{abstract}
        \noindent En este art\'iculo se expone la soluci\'on del problema B.Hydra del juez online \href{https://codeforces.com/} { \textcolor{blue}{CodeForce} }  
        si quiere ir directo al link del problema \href{https://codeforces.com/problemset/problem/243/B}{\textcolor{blue}{click aqui}} . En el art\'iculo se da una soluci\'on al problema , adem\'as de 
        demostrar la correctitud del algoritmo y la complejidad temporal del mismo . Al final del articulo se expone el c\'odigo en python de la soluci\'on del problema   
    \end{abstract}

    
    \section*{Orientaci\'on del Problema }
    \begin{flushleft}
        Un d\'ia Petya recibi\'o un regalo de cumplea\~nos de su mam\'a, un libro llamado "Leyes y mitos de la Teor\'ia de Grafos". De este libro Petya aprendi\'o acerca de los grafos Hydra.
        \\*
        Un grafo no dirigido es un Hydra; si este tiene una estructura como la mostrada en la figura de abajo.
        \\*
        Esta estructura tiene dos nodos $u$ y $v$ conectados por una arista. Estos son el pecho y el estomago de Hydra
        correspondientemente. El pecho es conectado con h nodos que son la cabeza de Hydra. El est\'omago est\'a conectado con t nodes que son la cola de Hydra.
        \\*
        Note que el Hydra es un \'arbol, que tiene $h+t+2$ nodos
    \end{flushleft}
    
    \begin{center}
        \begin{tikzpicture}[vertex/.style={draw,circle} ]
        \tikzstyle{stylenode} = [vertex ,fill = red ,draw = none ,text = black ,scale = 1]
            
            \node[ stylenode] (1) at (1.6,6) {} ; 
            \node[ stylenode] (2) at (1.4,5) {} ; 
            \node[ stylenode] (3) at (1.3,4) {} ; 
            \node[ stylenode] (4) at (3,5) {$u$} ; 

            \node[ stylenode] (5) at (7,4) {$v$} ; 
            \node[ stylenode] (6) at (9,5) {} ; 
            \node[ stylenode] (7) at (9,4) {} ; 
            \node[ stylenode] (8) at (9,3) {} ; 
            \node[ stylenode] (9) at (8.4,2) {} ; 
            


            \draw [line width = 2] (1) -- (4);
            \draw [line width = 2] (2) -- (4);
            \draw [line width = 2] (3) -- (4);

            \draw [line width = 2] (4) -- (5);

            \draw [line width = 2] (5) -- (6);
            \draw [line width = 2] (5) -- (7);
            \draw [line width = 2] (5) -- (8);
            \draw [line width = 2] (5) -- (9);

            \node[right = 0.5cm,text width = 4cm,font=\footnotesize,scale = 1.5] at (1)
            {
                heads
            };
            \node[right = 1.5cm,text width = 4cm,font=\footnotesize,scale = 1.5] at (4)
            {
                body
            };
            \node[left = 0.5cm,text width = 0.5cm,font=\footnotesize,scale = 1.5] at (6)
            {
                tails
            };

        \end{tikzpicture}
    \end{center}

    \begin{flushleft}
        Tambien, Petya tiene un grafo no dirigido G que consiste de n nodos y m aristas. Petya recibi\'o este como el \'ultimo regalo de cumplea\~nos de su mam\'a. El grafo G no contiene ni multiples aristas ni lazos.
    \end{flushleft}
  
    \begin{flushleft}
        Ahora Petya quiere encontrar un Hydra en el grafo G o estar segura que el grafo no tiene un Hydra.
    \end{flushleft}


    \vspace{1cm}
    \noindent \textcolor{black}{\Large Entrada }
    \begin{flushleft}
        La primera linea contine cuatro enteros n,m,h,t $\left(1\leq n,m\leq {10}^{5} , 1 \leq h , t,\leq 100\right)$, el n\'umero de nodos y las 
        aristas en el grafo \textit{G} y el n\'umero de la cabeza del Hydra y la cola.
        \\*
        Las proximas m lineas continen la descricion de las aristas del grafo \textit{G} . La $i-esima$ de esas lineas 
        continen dos enteros $a_i$ y $b_i$ , $\left(1 \leq a_i , b_i \leq n , a \neq b\right)$ el n\'umero de los nodos conectados por la $i-esima$ arista. 
        \\*
        Se garantiza que el grafo \textit{G} no contine lazos ni aristas multiples . Considera los nodos del grafo \textit{G} numerados desde 1 hasta n.  
    \end{flushleft}

    \vspace{0.5cm}
    \noindent \textcolor{black}{\Large Salida }
    \begin{flushleft}
        Si el grafo \textit{G} no contiene un hydra entonces imprimir -NO-.
        \\*
        En otro caso , en la primera linea imprimir -YES- . En la segunda linea imprimir dos enteros  - los n\'umeros de los nodos u y v. 
        En la tercera linea imprimir h n\'umeros  - los n\'umeros de los nodos que son la cabeza. En la cuarta linea imprimir t n\'umeros  - los n\'umeros de los nodos que son los pies. 
        Todos los n\'umeros que se impriman deben de ser distintos.
        \\*
        Si hay m\'ultiples posibles respuestas , tienes permitido imprimir cualquiera de ellas.   
    \end{flushleft}

    \noindent \textbf{\Large Ejemplo 1}
    \begin{minted}[bgcolor = bg , frame = single , framerule = 1pt , framesep = 5pt , gobble = 8 , label = INPUT] {console}
        9 12 2 3
        1 2
        2 3
        1 3
        1 4
        2 5
        4 5
        4 6
        6 5
        6 7
        7 5
        8 7
        9 1
    \end{minted}

    \begin{minted}[bgcolor = bg , frame = single , framerule = 1pt , framesep = 5pt , gobble = 8 , label = OUTPUT] {console}
        YES
        4 1
        5 6 
        9 3 2
    \end{minted}
    
    \vspace{1cm}
    \noindent \textbf{\Large Ejemplo 2}
    \begin{minted}[bgcolor = bg , frame = single , framerule = 1pt , framesep = 5pt , gobble = 8 , label = INPUT] {console}
        7 10 3 3
        1 2
        2 3
        1 3
        1 4
        2 5
        4 5
        4 6
        6 5
        6 7
        7 5
    \end{minted}
    
    \begin{minted}[bgcolor = bg , frame = single , framerule = 1pt , framesep = 5pt , gobble = 8 , label = OUTPUT] {console}
        NO
    \end{minted}

    \begin{algorithm}[h]
        \caption{Calculo de \textit{first hop}}
        \begin{algorithmic}[1]
       
          % ENTRADA / SALIDA
          \Require{ G , u , v } 
          \Ensure{head , tail}
       
          \State countcommun = 0 
          \ForAll {$a \in Ady \left[v\right] $} sjdfg\EndFor
       
          \For{$i;\ i++;\ |target|$}
            \State{traceroute($IP = IP[i],\ TTL_{initial} = 1$)}
          \EndFor
       
          \For{$i;\ i++;\ |target|$}
            \State{trace = read\_traceroute($IP[i]$)}
            \State{$TTL_{initial} = 1$}
            \For{hop \textbf{in} trace}
              \If{hop \textbf{in} $IP_{found}$}
                \State $TTL_{initial} ++$
              \Else
                 \State $IP_{found}$ \textbf{append} hop
                 \State $first\_hop[i] = TTL_{initial}$
                 \State \textbf{break}
              \EndIf
            \EndFor
          \EndFor
          \State{\textbf{return} $first\_hop$}
        \end{algorithmic}
      \end{algorithm}

    
    \section{An\'alisis del problema }

    \subsection{Dicuci\'on de lo que nos piden en el problema } 
    \noindent Tenemos que dado un grafo G no dirigido y dos entros h y t determinar si 
    existe alguna estuctura de la forma descrita en la orden del ejercicio. 
    Esto ser\'ia encontrar dos v\'ertices u , v adyacentes y 
    dos conjuntos independientes \textit{head} y \textit{tail } 
    de cardinalidad h y t respectivamente ,donde ni el v\'ertice u ni el v\'ertice v pueden estar 
    incluidos en los conjuntos \textit{tail} ni en el conjunto \textit{head}. 
    Tambi\'en se tiene que cumplir que todo los v\'ertice  del conjunto \textit{head} sean 
    adyacentes a u y todos los  v\'ertices del conjunto \textit{tail } sean adyacentes a v.

     

    \subsection*{Ideas para la resoluci\'on del problema }
    
    \noindent La primera idea que se me ocurri\'o para hacer esto fue recorrer todas 
    las aristas del grafo y ver si podia conformar un Hydra donde la arista que estaba 
    analizando era el Hydra del Grafo .
    \\*
    Luego de pensar el ejercicio de esta forma empece a analizar de que forma pod\'ia armar un Hydra 
    en el un grafo dado con la idea anterior  .

    \section{Soluci\'on del Problema }

    \noindent Voy a recorrer todas las aristas del grafo G y voy a analizar si 
    esta puede ser la arista que conforma el "cuerpo del hydra"  vamos a 
    denotarla arista e y a los extremos de la arista (los v\'ertices que une) 
    los vamos a llamar u , v para esto vamos a usar la notaci\'on  $u \leftrightarrow v$.
    Entonces el an\'alisis de una arista $e$ lo vamos  a hacer de la siguiente manera. 
    

    \subsection{An\'alisis de la arista e}

    \noindent Vamos a analizar la arista $e$ de forma tal que vamos a asumir que para nuestro an\'alisis la arista $e$ 
    es tomada en el siguiente sentido $u\xrightarrow[]{e}v$ y analizar si el extremo $u$ de la arista puede formar 
    la "cabeza" del Hydra (o sea que los v\'ertices adyacentes a $u$ que estoy buscando sean los que conformen el conjunto 
    \textit{head}) y el extremo $v$ sea el adyacente a los v\'ertices que conforman el \textit{tail} . Como el grafo es no 
    dirigido hay que analizar tambien el otro sentido de la arista , es decir considerar a $e$ de la forma $v\xrightarrow[]{e}u$ 

    \noindent Vamos a ir descartando el an\'alisis de algunas aristas y tratar de buscar la posible o no soluci\'on
    en otras que cumplan ciertos requisistos y para esto  voy a dividir el an\'alisis por casos

    \subsubsection{An\'alisis de casos } 

    \begin{itemize}
        \item \textit{\textbf {Caso 1 }} Si $deg (u) <  h +1$   o el $deg(v) < t+1 $ entonces no hay forma posible 
        de que esta arista forme parte del "cuerpo de hydra " . Es decir no puedo encontrar los conjuntos \textit{head} y 
        \textit{tail} de carinalidad h y t respectivamente porque $u$ y $v$ no tienen las los adyacentes necesarios para armar estos 
        conjuntos con las especificaciones que discutimos arriba 

        \item  \textit{\textbf {Caso 2}} Aqu\'i si llegamos a este an\'alisis es porque suponemos que el caso uno fue analizado. 
        Luego si el  \textit{Caso 1} no se cumple es porque las condiciones $deg (u) \geq h+1 $ y  $deg(v) \geq t+1$ se cumplen ambas a la vez.
        Teniendo en cuenta esto vamos ahora a analizar lo siguiente: 
        Si suponemos ahora que $deg (u) \geq h+t+1$  Entonces es posible  encontrar los subconjuntos independientes \textit{head}  y \textit{tail } de cardinalidad $h$ y $t$ respectivamente que
        nos piden en el ejercicio , pues en el peor de los escenarios  que es que todos los adyacentes de $v$ tambi\'en sean adyacentes de $u$ pero $u$ tiene suficientes adyacentes para particionar
        de forma tal que sea posible escoger los subconjuntos con las restricciones que analizamos arriba
        \\*
        En el caso en que no se cumpla que $deg (u) \geq h+t+1$ podemos analizar si $deg(v) \geq h+t+1$ , donde se puede formar de igual manera los conjuntos que andamos buscando 
        
        \item  \textit{\textbf{Caso 3}} Teniendo en cuenta que n\'inguno de los casos anteriores se cumpli\'o es 
        porque se cumple que $h+1 \leq deg(u) \leq h + t  $ y adem\'as se cumple que $t+1 \leq deg(v) \leq h + t  $
        luego habr\'ia que determinar aqui si existe o no la estructura que estamos buscando 
        \\*
        En este caso podemos comprobar la no existencia de un hidra de la siguiente manera :
        Si definimos a $U$ como el conjunto de v\'etices que son adyacentes a $u$ y a $v$ sin incluir a estos dos (porque si estamos analizando
        esto es porque esiste una arista que conecta a $u$ con $v$ ) , y en las restricciones del problema nos dicen 
        que $u$ y $v$ no pueden pertenecer a los conjunto  que estamos buscando, entonces podemos saber cu\'antos de estos v\'ertices
        que pertenecen a $U$ van a ser necesarios a la hora de conformar los conjuntos \textit{head} y \textit{tail} y si esta cantidad 
        es mayor que la cardinalidad del conjunto $U$ entonces no es posible  encontrar la estructura que queremos. 
    \end{itemize}


    \subsection{An\'alisis detallado del Caso 2}

    \noindent ?` La pregunta aqu\'i ser\'ia como formar los conjuntos que nos piden si se cumple que $deg (u) \geq h+t+1$  y que $deg(v) \geq t+1$ ? 
    La idea es recorrer todos los adyacentes de $v$ y escoger $t$ v\'ertices de estos para armar el conjunto \textit{tail} y marcarlos para saber
    que ya forman parte del conjunto \textit{tail}  , despues recorro los adyacentes de $u$ y escojo t de estos que no se hayan escogidos para el
    conjunto \textit{tail} , esto se puede hacer perfectamente pues $u$ tiene suficientes adyacentes para escoger y armar el \textit{head} una vez sea
    hayan escogidos los v\'ertices del \textit{tail} . 
    \\*
    Es importante aqu\'i hacerlo en el orden en que lo describ\'i (primero el \textit{tail} y despu\'es el \textit{head}), pues  si se hace en el otro orden 
    puede ocurrir que todos los que escog\'i para conformar el \textit{head} sean todos los comunes  o gran parte de ellos  (adyacentes a $u$ y a $v$ a la vez), 
    y despues cuando vaya aconformar el \textit{tail} no pueda hacerlo porque me faltan por escoger y los restantes ya hayan sido escogidos para conformar el \textit{head}. 
    Sin embargo esto no ocurre si primero escojo los vertices del \textit{tail } y despues los v\'ertices del \textit{head}
    \\*
    El mismo procedimiento se puede hacer para el caso en que $deg (v) \geq h+t+1$ y $deg(u) \geq h+1$ . Siempre tenemos que tener en cuenta primero escoger 
    (en este caso ) primero el \textit{head} y despues el \textit{tail} 
    

    \subsection{An\'alisis detallado del caso 3}

    \noindent Si Llegamos al caso 3 tenemos que se cumple que  $h+1 \leq deg(u) \leq h + t  $ y adem\'as se cumple que $t+1 \leq deg(v) \leq h + t  $
    \\*
    ?`Como saber si existe un Hydra en este caso o no ? 
    \\*
    Primero vamos a analizar un caso en el que no puede existir un Hydra . Y la idea es la siguiente: 
    Vamos a denotar a $C$ como el conjunto formado por los  adyacentes a $u$ que son tambi\'en adyacentes de $v$ .
    \\* 
    Vamos a denotar a $IndU$ como el conjunto de adyacentes de $u$ que no est\'an en $C$ , ($v \notin IndU$) 
    \\*
    Y vamos a denotar a $IndV$ como el conjunto de adyacente de $v$ que no est\'an en $C$ , ($u \notin IndV$)
    \\*
    Si se cumple que $ h -\vert IndU \vert  +  t - \vert IndV \vert > \vert C \vert $ entonces no se puede armar un Hydra . Porque la cantidad de v\'ertices necesarios 
    que tengo que escoger que son comunes para formar el \textit{tail } y el \textit{head} es mayor que la cantidad de v\'ertices que estos comparten en com\'un. 
    \\[5pt] 
    En caso que lo que analizamos antes no se cumpla entonces si podemos escoger el \textit{head} y el \textit{tail} 
    \\*
    ?` como hacemos esto ? 
    \\*
    Si grarantizamos que a la hora de escoger el conjunto \textit{tail} no escogemos una cantidad de  v\'ertices superior a los que son necesarios escoger del conjunto $C$ para formar el \textit{tail} entonces 
    estamos garantizando que vamos a poder despu\'es escoger el \textit{head} sin ning\'un impedimento , Igual pasa si escogemos primero los v\'ertices que conforman el conjunto \textit{head} 

    \subsection{Seudoc\'odigo del algoritmo de Dijkstra} 

    \begin{algorithm}[h]
        \caption{\textit{Dikjstra's algorithm}}
        \begin{algorithmic}[1]      
        \State $\lambda \left(s \right) \leftarrow 0 $
        \State $Q \leftarrow \{s\}$
        \State $P \leftarrow \emptyset $

        \While {$Q \neq \emptyset $}
            \State \textbf{choose a vertex $v \in Q $ for which $\lambda \left(v\right)$ is minimum}
            \State $Q$ $\leftarrow$ $Q$ $\backslash$ $\{v\}$
            \State $P$ $\leftarrow$ $P$ $\cup$  $\{v\}$
                \ForAll {$<u,v>$}
                    \If {$u \in Q $}
                            \State $\lambda $ $\left(u\right)$ $\leftarrow $ \textbf{min} $\{ \lambda \left(u\right) , \lambda \left(v\right) + l \left(e\right)\}$ 
                        \Else
                            \If {$u \notin P$} 
                                \State $\lambda\left(u\right)$ $ \leftarrow $ $ \lambda \left(v\right) + l \left(e\right)$
                                \State $ Q \leftarrow  Q \cup \{u\}$
                            \EndIf
                        \EndIf 
                    \EndFor
            \EndWhile 
            \end{algorithmic}
        \end{algorithm}

    
    
    \usemintedstyle{solarized-dark}
    \inputminted[bgcolor = bg ,frame = single , framerule = 3pt, framesep = 5pt , linenos = true ]{python} {../hydra.py}
    
\end{document}